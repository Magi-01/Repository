\documentclass{article}%
\usepackage[T1]{fontenc}%
\usepackage[utf8]{inputenc}%
\usepackage{lmodern}%
\usepackage{textcomp}%
\usepackage{lastpage}%
\usepackage{listings}%
\usepackage{xcolor}%
\usepackage{hyperref}%
\usepackage{longtable}%
\usepackage{tabularx}%
%
\title{SQL Database Design and Queries for Dimensional Transfer Game}%
\author{}%
%
\begin{document}%
\normalsize%

\lstset{
    basicstyle=\ttfamily\footnotesize,
    numbers=left,
    numberstyle=\tiny\color{gray},
    keywordstyle=\color{blue},
    commentstyle=\color{green},
    stringstyle=\color{red},
    breaklines=true,
    showstringspaces=false,
    frame=single,
    captionpos=b
}
%
\maketitle%
\section{Project Description}%
\label{sec:ProjectDescription}%
The project involves the creation of a database schema for a game. The database will track various entities such as players, non{-}player characters (NPCs), quests, items, achievements, guilds, and dimensions. It will also record relationships and actions such as quest completion, item ownership, and guild membership.

%
\section{SQL Operations}%
\label{sec:SQLOperations}%
\subsection{View Player Stats}%
\label{subsec:ViewPlayerStats}%

        This procedure displays the basic stats of a specific player, including their name, level, experience, and player score.
        \begin{lstlisting}[language=sql,caption=View Player Stats]
        DELIMITER //
        CREATE PROCEDURE ViewPlayerStats(IN playerID INT)
        BEGIN
            SELECT name, level, experience, player_score
            FROM Player
            WHERE player_id = playerID;
        END //
        DELIMITER ;
        \end{lstlisting}
        

%
\subsection{View Inventory}%
\label{subsec:ViewInventory}%

        This procedure lists all items in a player's inventory, showing item name, quantity, type, and value.
        \begin{lstlisting}[language=sql,caption=View Inventory]
        DELIMITER //
        CREATE PROCEDURE ViewInventory(IN playerID INT)
        BEGIN
            SELECT Item.name AS ItemName, Player_Item.quantity, Item.type, Item.value
            FROM Player_Item
            INNER JOIN Item ON Player_Item.item_id = Item.item_id
            WHERE Player_Item.player_id = playerID;
        END //
        DELIMITER ;
        \end{lstlisting}
        

%
\section{Schema Concettuale}%
\label{sec:SchemaConcettuale}%
\begin{tabularx}{\textwidth}{|l|l|l|}%
\hline%
\textbf{Entity 1}&\textbf{Relationship}&\textbf{Entity 2}\\%
\hline%
Player (1:N)&Belong&Guild (1:1)\\%
\hline%
Player (1:N)&Complete&Quest (1:N)\\%
\hline%
Player (1:N)&Own&Player\_Item (1:N)\\%
\hline%
Player\_Item (1:N)&Legal\_item&Item (1:N)\\%
\hline%
NPC (1:1)&Affiliation&Guild (1:1)\\%
\hline%
Dimension (1:1)&Complete&Quest (1:N)\\%
\hline%
\end{tabularx}

%
\section{Schema Logico}%
\label{sec:SchemaLogico}%
\subsection{Entità}%
\label{subsec:Entit}%
\begin{longtable}{|l|l|l|l|l|}%
\hline%
\textbf{Entità}&\textbf{Descrizione}&\textbf{Attributi 1}&\textbf{Attributi 2}&\textbf{Attributi 3}\\%
\hline%
Player&A user of the game&\underline{player\_id}&name&level\\%
&experience&\underline{[guild\_id]}&\underline{[quest\_id]}&\\%
&player\_score&&&\\%
\hline%
NPC&Non{-}player character&\underline{npc\_id}&name&role\\%
&alignment&\underline{[guild\_id]}&&\\%
\hline%
\end{longtable}

%
\subsection{Relazioni}%
\label{subsec:Relazioni}%
\begin{longtable}{|l|l|l|l|l|}%
\hline%
\textbf{Relazioni}&\textbf{Descrizione}&\textbf{Attributi 1}&\textbf{Attributi 2}&\textbf{Attributi 3}\\%
\hline%
Completion&Record of completed quests&\underline{[player\_id]}&\underline{[quest\_id]}&state\\%
\hline%
\end{longtable}

%
\section{Redundancy Analysis}%
\label{sec:RedundancyAnalysis}%
Redundancy in the unnormalized schema can lead to data anomalies and inefficiencies. Detailed analysis of redundancy is as follows:%
\begin{itemize}%
\item%
\textbf{Player}: Contains redundant attributes \underline{guild\_name} and \underline{quest\_name}.%
\item%
\textbf{NPC}: Contains a redundant attribute \underline{guild\_name}.%
\item%
\textbf{Player\_Item}: Contains a redundant attribute \underline{item\_condition}.%
\item%
\textbf{Achievement}: Contains a redundant attribute \underline{achievement\_status}.%
\item%
\textbf{Guild}: Contains redundant attributes \underline{guild\_leader} and \underline{guild\_points}.%
\end{itemize}

%
\section{Restructuring with Analysis of Redundancy and Eventual Additions/Removals}%
\label{sec:RestructuringwithAnalysisofRedundancyandEventualAdditions/Removals}%
In this section, we remove redundancy from the schema. A derived attribute is one that can be calculated or inferred from other attributes in the database. We will remove such attributes and show the updated schema.%
\subsection{Removal of Redundancy}%
\label{subsec:RemovalofRedundancy}%
By removing the redundant attributes, the schema is optimized to avoid data anomalies and inefficiencies. Here is the detailed description of the changes made:%
\begin{itemize}%
\item%
\textbf{Player}: The attributes \underline{guild\_name} and \underline{quest\_name} were removed. These attributes can be derived from \underline{guild\_id} and \underline{quest\_id}, respectively.%
\item%
\textbf{NPC}: The attribute \underline{guild\_name} was removed because it can be derived from \underline{guild\_id}.%
\item%
\textbf{Player\_Item}: The attribute \underline{item\_condition} was removed because it is a derived or calculated attribute.%
\item%
\textbf{Achievement}: The attribute \underline{achievement\_status} was removed because it can be inferred from \underline{date\_earned}.%
\item%
\textbf{Guild}: The attributes \underline{guild\_leader} and \underline{guild\_points} were removed because they may be derived or unnecessary depending on the use case.%
\end{itemize}

%
\subsection{Output after Redundancy Removal}%
\label{subsec:OutputafterRedundancyRemoval}%
The resulting entities and relationships after removing redundancy are:%
\subsection{Entities}%
\label{subsec:Entities}%
\begin{itemize}%
\item%
\textbf{Player}: \texttt{player\_id, name, level, experience, guild\_id, quest\_id, player\_score}%
\item%
\textbf{NPC}: \texttt{npc\_id, name, role, alignment, guild\_id}%
\item%
\textbf{Quest}: \texttt{quest\_id, name, description, reward}%
\item%
\textbf{Player\_Item}: \texttt{player\_item\_id, player\_id, item\_id, quantity}%
\item%
\textbf{Item}: \texttt{item\_id, name, type, value}%
\item%
\textbf{Achievement}: \texttt{achievement\_id, name, description, date\_earned}%
\item%
\textbf{Guild}: \texttt{guild\_id, name, alignment}%
\item%
\textbf{Dimension}: \texttt{dimension\_id, name, description, difficulty\_level}%
\end{itemize}

%
\subsection{Relationships}%
\label{subsec:Relationships}%
\begin{itemize}%
\item%
\textbf{Completion}: \texttt{player\_id, quest\_id, state}%
\item%
\textbf{Belong}: \texttt{player\_id, guild\_id}%
\item%
\textbf{Own}: \texttt{player\_item\_id, player\_id, item\_id, quantity}%
\item%
\textbf{Affiliation}: \texttt{npc\_id, guild\_id}%
\end{itemize}

%
\end{document}