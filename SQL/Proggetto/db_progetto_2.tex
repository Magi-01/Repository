\documentclass{article}
\usepackage{geometry}
\usepackage{longtable}
\usepackage{array}
\usepackage{lipsum}
\usepackage{listings}
\usepackage{graphicx}

\geometry{a4paper, margin=1in}

\title{DataBase Project}
\author{Mutua Fadhla Mohamed | SM3201434}
\date{\today}

\begin{document}

\maketitle

\section{Description of what the database does}

This database manages a game world where players, NPCs (non-player characters), and quests interact. 
Players can join guilds, complete quests, and earn achievements that unlock dimensions. 
NPCs can belong to guilds and initiate quests. The database ensures data integrity and proper relationships 
between entities using triggers and constraints.

\section{Queries}

\subsection{Triggers}

\subsection{lock\_achievements\_if\_invalid\_item}

\texttt{General Description:} This trigger locks all achievements for a player if any of the player's items
become invalid (state is false).

\begin{lstlisting}[language=SQL]
CREATE TRIGGER lock_achievements_if_invalid_item
AFTER UPDATE ON Player_Item
FOR EACH ROW
BEGIN
    DECLARE player_achievements_locked BOOLEAN DEFAULT FALSE;

    -- Check if any item for the player is invalid
    IF EXISTS (SELECT 1 FROM Player_Item
        WHERE player_id = NEW.player_id AND state = FALSE) THEN
        SET player_achievements_locked = TRUE;
    END IF;

    -- Create a temporary table to hold achievement IDs
    CREATE TEMPORARY TABLE Temp_Achievements AS
    SELECT ca.achievement_id
    FROM Check_Achievement ca
    JOIN Complete c ON c.quest_id = ca.quest_id
    WHERE c.player_id = NEW.player_id;

    -- Update the achievement status based on item state
    UPDATE Check_Achievement
    SET requires_all_player_items = NOT player_achievements_locked
    WHERE achievement_id IN (SELECT achievement_id FROM Temp_Achievements);

    -- Drop the temporary table
    DROP TEMPORARY TABLE Temp_Achievements;
END;
\end{lstlisting}

\texttt{Line-by-Line Explanation:}
\begin{itemize}
    \item \lstinline|CREATE TRIGGER lock_achievements_if_invalid_item|: Defines a new trigger named \linebreak \lstinline|lock_achievements_if_invalid_item|.
    \item \lstinline|AFTER UPDATE ON Player_Item|: Specifies that this trigger will be executed after \linebreak an update operation on the \lstinline|Player_Item| table.
    \item \lstinline|FOR EACH ROW|: Indicates that the trigger will be executed for each row that is being updated.
    \item \lstinline|BEGIN|: Starts the block of SQL statements that make up the trigger.
    \item \lstinline|DECLARE player_achievements_locked BOOLEAN DEFAULT FALSE|: Declares a variable \linebreak \lstinline|player_achievements_locked| and initializes it to \lstinline|FALSE|.
    \item \lstinline|IF EXISTS (SELECT 1 FROM Player_Item| \linebreak \lstinline|WHERE player_id = NEW.player_id AND state = FALSE) THEN|: Checks if any item for the player is invalid (state is false).
    \item \lstinline|SET player_achievements_locked = TRUE|: If any invalid item is found, set the variable \linebreak \lstinline|player_achievements_locked| to \lstinline|TRUE|.
    \item \lstinline|CREATE TEMPORARY TABLE Temp_Achievements AS SELECT ca.achievement_id| \linebreak \lstinline|FROM Check_Achievement ca JOIN Complete c ON c.quest_id = ca.quest_id| \linebreak \lstinline|WHERE c.player_id = NEW.player_id|: Creates a temporary table to hold achievement IDs for the player.
    \item \lstinline|UPDATE Check_Achievement SET requires_all_player_items = NOT player_achievements_locked| \linebreak \lstinline|WHERE achievement_id IN (SELECT achievement_id FROM Temp_Achievements)|: \linebreak Updates the achievement status based on the item state.
    \item \lstinline|DROP TEMPORARY TABLE Temp_Achievements|: Drops the temporary table.
    \item \lstinline|END|: Ends the trigger.
\end{itemize}

\subsection{unlock\_dimension}

\texttt{General Description:} This trigger unlocks dimensions for a player if all their achievements are true.

\begin{lstlisting}[language=SQL]
CREATE TRIGGER unlock_dimension
AFTER UPDATE ON Complete
FOR EACH ROW
BEGIN
    DECLARE all_achievements_true BOOLEAN;
    DECLARE player_id INT;
    SET player_id = NEW.player_id;

    SELECT COUNT(*) = 0 INTO all_achievements_true
    FROM Achievement
    JOIN Check_Achievement ON Achievement.achievement_id = Check_Achievement.achievement_id
    WHERE Check_Achievement.requires_all_player_items = TRUE
    AND Check_Achievement.quest_id NOT IN (
        SELECT quest_id
        FROM Complete
        WHERE player_id = player_id
    );

    IF all_achievements_true THEN
        INSERT IGNORE INTO Travel (player_id, dimension_id)
        SELECT player_id, dimension_id
        FROM Unlocks;
    END IF;
END;
\end{lstlisting}

\texttt{Line-by-Line Explanation:}
\begin{itemize}
    \item \lstinline|CREATE TRIGGER unlock_dimension|: Defines a new trigger named \linebreak \lstinline|unlock_dimension|.
    \item \lstinline|AFTER UPDATE ON Complete|: Specifies that this trigger will be executed after an update operation on the \lstinline|Complete| table.
    \item \lstinline|FOR EACH ROW|: Indicates that the trigger will be executed for each row that is being updated.
    \item \lstinline|BEGIN|: Starts the block of SQL statements that make up the trigger.
    \item \lstinline|DECLARE all_achievements_true BOOLEAN|: Declares a variable \lstinline|all_achievements_true|.
    \item \lstinline|DECLARE player_id INT|: Declares a variable \lstinline|player_id|.
    \item \lstinline|SET player_id = NEW.player_id|: Sets the \lstinline|player_id| to the ID of the player from the updated row.
    \item \lstinline|SELECT COUNT(*) = 0 INTO all_achievements_true FROM Achievement| \linebreak \lstinline|JOIN Check_Achievement ON Achievement.achievement_id = Check_Achievement.achievement_id| \linebreak \lstinline|WHERE Check_Achievement.requires_all_player_items = TRUE AND| \linebreak \lstinline|Check_Achievement.quest_id NOT IN (SELECT quest_id FROM Complete| \linebreak \lstinline|WHERE player_id = player_id)|: \linebreak \lstinline|Checks if all achievements are true for the player.|
    \item \lstinline|IF all_achievements_true THEN|: If all achievements are true, then proceed.
    \item \lstinline|INSERT IGNORE INTO Travel (player_id, dimension_id)| \linebreak \lstinline|SELECT player_id, dimension_id FROM Unlocks|: Inserts records into the \lstinline|Travel| table to unlock dimensions for the player.
    \item \lstinline|END IF|: Ends the conditional statement.
    \item \lstinline|END|: Ends the trigger.
\end{itemize}

\subsection{assign\_quests\_after\_talk}

\texttt{General Description:} This trigger assigns quests to a player based on their guild affiliation after talking to an NPC.

\begin{lstlisting}[language=SQL]
CREATE TRIGGER assign_quests_after_talk
AFTER INSERT ON Talks
FOR EACH ROW
BEGIN
    DECLARE npc_guild_id INT;
    DECLARE player_guild_id INT;

    SELECT guild_id INTO npc_guild_id FROM NPC WHERE npc_id = NEW.npc_id;
    SELECT guild_id INTO player_guild_id FROM Player WHERE player_id = NEW.player_id;

    -- If NPC has no guild affiliation and player has no guild, assign quest
    IF npc_guild_id IS NULL AND player_guild_id IS NULL THEN
        INSERT INTO Complete (player_id, quest_id)
        SELECT NEW.player_id, quest_id
        FROM Initiate
        WHERE npc_id = NEW.npc_id
        LIMIT 1;
    END IF;

    -- If NPC and player have the same guild affiliation, assign quest
    IF npc_guild_id = player_guild_id THEN
        INSERT INTO Complete (player_id, quest_id)
        SELECT NEW.player_id, quest_id
        FROM Initiate
        WHERE npc_id = NEW.npc_id
        LIMIT 1;
    END IF;
END;
\end{lstlisting}

\texttt{Line-by-Line Explanation:}
\begin{itemize}
    \item \lstinline|CREATE TRIGGER assign_quests_after_talk|: Defines a new trigger named \linebreak \lstinline|assign_quests_after_talk|.
    \item \lstinline|AFTER INSERT ON Talks|: Specifies that this trigger will be executed after an insert operation on the \lstinline|Talks| table.
    \item \lstinline|FOR EACH ROW|: Indicates that the trigger will be executed for each row that is being inserted.
    \item \lstinline|BEGIN|: Starts the block of SQL statements that make up the trigger.
    \item \lstinline|DECLARE npc_guild_id INT|: Declares a variable \lstinline|npc_guild_id|.
    \item \lstinline|DECLARE player_guild_id INT|: Declares a variable \lstinline|player_guild_id|.
    \item \lstinline|SELECT guild_id INTO npc_guild_id FROM NPC WHERE npc_id = NEW.npc_id|: Retrieves the guild ID of the NPC and stores it in \lstinline|npc_guild_id|.
    \item \lstinline|SELECT guild_id INTO player_guild_id FROM Player WHERE player_id = NEW.player_id|: Retrieves the guild ID of the player and stores it in \lstinline|player_guild_id|.
    \item \lstinline|IF npc_guild_id IS NULL AND player_guild_id IS NULL THEN|: Checks if both the NPC and player have no guild affiliation.
    \item \lstinline|INSERT INTO Complete (player_id, quest_id)| \linebreak \lstinline|SELECT NEW.player_id, quest_id FROM Initiate WHERE npc_id = NEW.npc_id| \linebreak \lstinline|LIMIT 1|: Assigns a quest to the player.
    \item \lstinline|END IF|: Ends the conditional statement.
    \item \lstinline|IF npc_guild_id = player_guild_id THEN|: Checks if the NPC and player have the same guild affiliation.
    \item \lstinline|INSERT INTO Complete (player_id, quest_id)| \linebreak \lstinline|SELECT NEW.player_id, quest_id FROM Initiate WHERE npc_id = NEW.npc_id| \linebreak \lstinline|LIMIT 1|: Assigns a quest to the player.
    \item \lstinline|END IF|: Ends the conditional statement.
    \item \lstinline|END|: Ends the trigger.
\end{itemize}

\subsection{Queries}

\subsection{Essential Queries}

\subsubsection{Test\_UpdatePlayerItem}

\texttt{General Description:} This procedure tests updating player items to see if they remain true.

\begin{lstlisting}[language=SQL]
CREATE PROCEDURE Test_UpdatePlayerItem()
BEGIN
    UPDATE Player_Item
    SET state = TRUE
    WHERE player_item_id IN (SELECT player_item_id FROM Player_Item LIMIT 5);
END;
\end{lstlisting}

\texttt{Line-by-Line Explanation:}
\begin{itemize}
    \item \lstinline|CREATE PROCEDURE Test_UpdatePlayerItem()|: Defines a stored procedure named \linebreak \lstinline|Test_UpdatePlayerItem|.
    \item \lstinline|BEGIN|: Starts the block of SQL statements that make up the procedure.
    \item \lstinline|UPDATE Player_Item SET state = TRUE WHERE player_item_id IN| \linebreak \lstinline|(SELECT player_item_id FROM Player_Item LIMIT 5)|: Updates the state of the first five player items to \lstinline|TRUE|.
    \item \lstinline|END|: Ends the procedure.
\end{itemize}

\subsubsection{Test\_InsertQuestCompletion}

\texttt{General Description:} This procedure tests inserting quest completion records.

\begin{lstlisting}[language=SQL]
CREATE PROCEDURE Test_InsertQuestCompletion()
BEGIN
    INSERT INTO Complete (player_id, quest_id)
    VALUES (1, 1), (2, 2), (3, 3), (4, 4), (5, 5);
END;
\end{lstlisting}

\texttt{Line-by-Line Explanation:}
\begin{itemize}
    \item \lstinline|CREATE PROCEDURE Test_InsertQuestCompletion()|: Defines a stored procedure named \linebreak \lstinline|Test_InsertQuestCompletion|.
    \item \lstinline|BEGIN|: Starts the block of SQL statements that make up the procedure.
    \item \lstinline|INSERT INTO Complete (player_id, quest_id) VALUES (1, 1), (2, 2), (3, 3),| \linebreak \lstinline|(4, 4), (5, 5)|: Inserts completion records for the specified player and quest IDs.
    \item \lstinline|END|: Ends the procedure.
\end{itemize}

\subsubsection{ViewAllPlayerItemsDetailed}

\texttt{General Description:} This query retrieves detailed information about all player items.

\begin{lstlisting}[language=SQL]
CREATE PROCEDURE ViewAllPlayerItemsDetailed()
BEGIN
    SELECT pi.player_item_id, pi.player_id, pi.item_id, i.item_name, pi.state
    FROM Player_Item pi
    JOIN Item i ON pi.item_id = i.item_id;
END;
\end{lstlisting}

\texttt{Line-by-Line Explanation:}
\begin{itemize}
    \item \lstinline|CREATE PROCEDURE ViewAllPlayerItemsDetailed()|: Defines a stored procedure named \linebreak \lstinline|ViewAllPlayerItemsDetailed|.
    \item \lstinline|BEGIN|: Starts the block of SQL statements that make up the procedure.
    \item \lstinline|SELECT pi.player_item_id, pi.player_id, pi.item_id, i.item_name, pi.state| \linebreak \lstinline|FROM Player_Item pi JOIN Item i ON pi.item_id = i.item_id|: Selects detailed information about all player items, including item names and states.
    \item \lstinline|END|: Ends the procedure.
\end{itemize}

\subsubsection{ViewAllQuestsDetailed}

\texttt{General Description:} This query retrieves detailed information about all quests.

\begin{lstlisting}[language=SQL]
CREATE PROCEDURE ViewAllQuestsDetailed()
BEGIN
    SELECT quest_id, quest_name, state
    FROM Quest;
END;
\end{lstlisting}

\texttt{Line-by-Line Explanation:}
\begin{itemize}
    \item \lstinline|CREATE PROCEDURE ViewAllQuestsDetailed()|: Defines a stored procedure named \linebreak \lstinline|ViewAllQuestsDetailed|.
    \item \lstinline|BEGIN|: Starts the block of SQL statements that make up the procedure.
    \item \lstinline|SELECT quest_id, quest_name, state FROM Quest|: Selects detailed information about all quests, including their states.
    \item \lstinline|END|: Ends the procedure.
\end{itemize}

\subsubsection{ViewAllCompletedQuestsDetailed}

\texttt{General Description:} This query retrieves detailed information about all completed quests.

\begin{lstlisting}[language=SQL]
CREATE PROCEDURE ViewAllCompletedQuestsDetailed()
BEGIN
    SELECT c.player_id, p.player_name, q.quest_name, q.state
    FROM Complete c
    JOIN Player p ON c.player_id = p.player_id
    JOIN Quest q ON c.quest_id = q.quest_id
    ORDER BY p.player_id, q.quest_name;
END;
\end{lstlisting}

\texttt{Line-by-Line Explanation:}
\begin{itemize}
    \item \lstinline|CREATE PROCEDURE ViewAllCompletedQuestsDetailed()|: Defines a stored procedure named \linebreak \lstinline|ViewAllCompletedQuestsDetailed|.
    \item \lstinline|BEGIN|: Starts the block of SQL statements that make up the procedure.
    \item \lstinline|SELECT c.player_id, p.player_name, q.quest_name, q.state FROM Complete c| \linebreak \lstinline|JOIN Player p ON c.player_id = p.player_id| \linebreak \lstinline|JOIN Quest q ON c.quest_id = q.quest_id| \linebreak \lstinline|ORDER BY p.player_id, q.quest_name|: Selects detailed information about all completed quests, including player names and quest states.
    \item \lstinline|END|: Ends the procedure.
\end{itemize}

\begin{itemize}
    \item ViewNPCGuilds: Retrieves and displays NPCs along with their guild names, defaulting to an empty string if they do not belong to a guild.
\end{itemize}

\section{Conceptual schema}

\subsection{Entities and Attributes}

\begin{longtable}{|>{\raggedright}m{0.3\textwidth}|>{\raggedright\arraybackslash}m{0.6\textwidth}|}
\hline
\textbf{Entity} & \textbf{Description} \\
\hline
\endfirsthead
\multicolumn{2}{c}{{\bfseries \tablename\ \thetable{} -- continued from previous page}} \\
\hline
\textbf{Entity} & \textbf{Description} \\
\hline
\endhead
\hline \multicolumn{2}{|r|}{{Continued on next page}} \\ \hline
\endfoot
\hline
\endlastfoot
Player & Represents players in the game. \\
\hline
NPC & Represents non-player characters. \\
\hline
Guild & Represents guilds. \\
\hline
Quest & Represents quests. \\
\hline
Achievement & Represents achievements. \\
\hline
Dimension & Represents dimensions. \\
\hline
Item & Represents items. \\
\hline
Player\_Item & Represents items owned by players. \\
\hline
\end{longtable}

\begin{figure}[ht]
    \centering
    \includegraphics[width=\textwidth]{C:/Users/mutua/Documents/Repository/SQL/Proggetto/vcs.png}
    \caption{Conceptual schema}
    \label{fig:conceptual_schema}
\end{figure}

\subsection{Relationships}

\subsection{Relationship Descriptions}

\begin{itemize}
    \item \lstinline|player-complete-quest|: This relationship indicates that a player has completed a specific quest.
    \item \lstinline|player-talks-npc|: This relationship records interactions between players and NPCs.
    \item \lstinline|player-belong-guild|: This relationship denotes which guild a player belongs to.
    \item \lstinline|player-own-player_item|: This relationship indicates which items are owned by a player.
    \item \lstinline|player-travel-dimension|: This relationship indicates the dimensions a player has traveled to.
    \item \lstinline|guild-affiliated-npc|: This relationship indicates the affiliation between guilds and NPCs.
    \item \lstinline|npc-initiate-quest|: This relationship denotes the quests initiated by NPCs.
    \item \lstinline|quest-check-achievement|: This relationship indicates the quests that are required for achieving specific achievements.
    \item \lstinline|quest-check-player_item|: This relationship indicates the quests that require specific player items.
    \item \lstinline|player_item-check-achievement|: This relationship indicates which player items are checked for achievements.
    \item \lstinline|player_item-refers-item|: This relationship denotes which items are referred to by player items.
    \item \lstinline|achievement-unlocks-dimension|: This relationship indicates which achievements unlock specific dimensions.
\end{itemize}

\begin{longtable}{|>{\raggedright}m{0.3\textwidth}|>{\centering}m{0.3\textwidth}|>{\raggedright\arraybackslash}m{0.3\textwidth}|}
\hline
\textbf{Entity (0/1;1/n)} & \textbf{Relationship} & \textbf{Entity (0/1;1/n)} \\
\hline
\endfirsthead
\multicolumn{3}{c}{{\bfseries \tablename\ \thetable{} -- continued from previous page}} \\
\hline
\textbf{Entity (0/1;1/n)} & \textbf{Relationship} & \textbf{Entity (0/1;1/n)} \\
\hline
\endhead
\hline \multicolumn{3}{|r|}{{Continued on next page}} \\ \hline
\endfoot
\hline
\endlastfoot
Player (0/1) & Complete & Quest (1/n) \\
\hline
Player (0/1) & Talks & NPC (1/n) \\
\hline
Player (0/n) & Belong & Guild (1/1) \\
\hline
Player (0/1) & Own & Player\_Item (0/n) \\
\hline
Player (0/1) & Travel & Dimension (1/n) \\
\hline
Guild (0/1) & Affiliated & NPC (1/n) \\
\hline
NPC (0/1) & Initiate & Quest (1/n) \\
\hline
Quest (0/1) & Check Achievement & Achievement (1/n) \\
\hline
Quest (0/1) & Check Player\_Item & Player\_Item (1/n) \\
\hline
Player\_Item (0/1) & Check Achievement & Achievement (1/n) \\
\hline
Player\_Item (0/1) & Refers & Item (1/n) \\
\hline
Achievement (0/1) & Unlocks & Dimension (1/n) \\
\hline
\end{longtable}

\section{Detailed Redundancy Analysis}

\begin{itemize}
    \item \textbf{Player\_Item and Quest:}
    \begin{itemize}
        \item \textbf{Redundancy in Checking Items and Achievements:}
        \begin{itemize}
            \item Quest table includes attributes to check player items and achievements.
            \item Player\_Item also checks achievements, leading to redundancy.
            \item Separate tables are used to check relationships between quests, items, and achievements, causing overlapping data.
        \end{itemize}
        \item \textbf{Overlapping Attributes:}
        \begin{itemize}
            \item Quest and Check\_Achievement both reference achievement requirements for quests.
            \item Player\_Item and Check\_Achievement both reference player item checks.
        \end{itemize}
    \end{itemize}
    \item \textbf{Guild and NPC:}
    \begin{itemize}
        \item \textbf{Redundant Relationships:}
        \begin{itemize}
            \item Guild-Affiliated-NPC and Player-Belong-Guild relationships overlap.
            \item Information about guild affiliation is duplicated.
        \end{itemize}
    \end{itemize}
\end{itemize}

\section{Removal/addition caused by redundancy analysis}

\begin{itemize}
    \item Removal of Overlapping Attributes:
    \begin{itemize}
        \item Merged the relationships between Quest, Player\_Item, and Check\_Achievement into a unified structure.
    \end{itemize}
    \item Simplification of Relationships:
    \begin{itemize}
        \item Unified guild affiliation information to avoid redundancy.
    \end{itemize}
\end{itemize}

\subsection{Resulting Conceptual Schema}

\begin{longtable}{|>{\raggedright}m{0.3\textwidth}|>{\raggedright\arraybackslash}m{0.6\textwidth}|}
\hline
\textbf{Entity} & \textbf{Description} \\
\hline
\endfirsthead
\multicolumn{2}{c}{{\bfseries \tablename\ \thetable{} -- continued from previous page}} \\
\hline
\textbf{Entity} & \textbf{Description} \\
\hline
\endhead
\hline \multicolumn{2}{|r|}{{Continued on next page}} \\ \hline
\endfoot
\hline
\endlastfoot
Player & Represents players in the game. \\
\hline
NPC & Represents non-player characters. \\
\hline
Guild & Represents guilds. \\
\hline
Quest & Represents quests. \\
\hline
Achievement & Represents achievements. \\
\hline
Dimension & Represents dimensions. \\
\hline
Item & Represents items. \\
\hline
Player\_Item & Represents items owned by players. \\
\hline
\end{longtable}

\section{Logical schema}

\subsection{Entities and Attributes}

\begin{longtable}{|>{\raggedright}m{0.3\textwidth}|>{\raggedright\arraybackslash}m{0.6\textwidth}|}
\hline
\textbf{Entity} & \textbf{Attributes} \\
\hline
\endfirsthead
\multicolumn{2}{c}{{\bfseries \tablename\ \thetable{} -- continued from previous page}} \\
\hline
\textbf{Entity} & \textbf{Attributes} \\
\hline
\endhead
\hline \multicolumn{2}{|r|}{{Continued on next page}} \\ \hline
\endfoot
\hline
\endlastfoot
Player & player\_id (INT, PK), player\_name (VARCHAR), guild\_id (INT, FK) \\
\hline
NPC & npc\_id (INT, PK), npc\_name (VARCHAR), guild\_id (INT, FK) \\
\hline
Guild & guild\_id (INT, PK), guild\_name (VARCHAR) \\
\hline
Quest & quest\_id (INT, PK), quest\_name (VARCHAR), state (BOOLEAN) \\
\hline
Achievement & achievement\_id (INT, PK), achievement\_name (VARCHAR) \\
\hline
Dimension & dimension\_id (INT, PK), dimension\_name (VARCHAR) \\
\hline
Item & item\_id (INT, PK), item\_name (VARCHAR) \\
\hline
Player\_Item & player\_item\_id (INT, PK), player\_id (INT, FK), item\_id (INT, FK), state (BOOLEAN) \\
\hline
\end{longtable}

\subsection{Relationships}

\begin{longtable}{|>{\raggedright}m{0.3\textwidth}|>{\raggedright\arraybackslash}m{0.7\textwidth}|}
\hline
\textbf{Relationship} & \textbf{Attributes} \\
\hline
\endfirsthead
\multicolumn{2}{c}{{\bfseries \tablename\ \thetable{} -- continued from previous page}} \\
\hline
\textbf{Relationship} & \textbf{Attributes} \\
\hline
\endhead
\hline \multicolumn{2}{|r|}{{Continued on next page}} \\ \hline
\endfoot
\hline
\endlastfoot
Complete & player\_id (INT, FK), quest\_id (INT, FK) \\
\hline
Talks & player\_id (INT, FK), npc\_id (INT, FK) \\
\hline
Belong & player\_id (INT, FK), guild\_id (INT, FK) \\
\hline
Own & player\_id (INT, FK), player\_item\_id (INT, FK) \\
\hline
Travel & player\_id (INT, FK), dimension\_id (INT, FK) \\
\hline
Affiliated & guild\_id (INT, FK), npc\_id (INT, FK), affiliation (VARCHAR) \\
\hline
Initiate & npc\_id (INT, FK), quest\_id (INT, FK) \\
\hline
Check\_Achievement & quest\_id (INT, FK), achievement\_id (INT, FK), requires\_all\_player\_items (BOOLEAN) \\
\hline
Check\_Player\_Item & quest\_id (INT, FK), player\_item\_id (INT, FK) \\
\hline
Refers & player\_item\_id (INT, FK), item\_id (INT, FK) \\
\hline
Unlock & achievement\_id (INT, FK), dimension\_id (INT, FK) \\
\hline
\end{longtable}

\section{Normalization of logical schema}

\subsection{First Normal Form (1NF)}

\begin{itemize}
    \item Player: Each player has a unique player\_id, player\_name, and guild\_id.
    \item NPC: Each NPC has a unique npc\_id, npc\_name, and guild\_id.
    \item Guild: Each guild has a unique guild\_id and guild\_name.
    \item Quest: Each quest has a unique quest\_id, quest\_name, and state.
    \item Achievement: Each achievement has a unique achievement\_id and achievement\_name.
    \item Dimension: Each dimension has a unique dimension\_id and dimension\_name.
    \item Item: Each item has a unique item\_id and item\_name.
    \item Player\_Item: Each player item has a unique player\_item\_id, player\_id, item\_id, and state.
\end{itemize}

\subsection{Second Normal Form (2NF)}

\begin{itemize}
    \item Player: player\_name and guild\_id are fully dependent on player\_id.
    \item NPC: npc\_name and guild\_id are fully dependent on npc\_id.
    \item Guild: guild\_name is fully dependent on guild\_id.
    \item Quest: quest\_name and state are fully dependent on quest\_id.
    \item Achievement: achievement\_name is fully dependent on achievement\_id.
    \item Dimension: dimension\_name is fully dependent on dimension\_id.
    \item Item: item\_name is fully dependent on item\_id.
    \item Player\_Item: player\_id, item\_id, and state are fully dependent on player\_item\_id.
\end{itemize}

\subsection{Third Normal Form (3NF)}

\begin{itemize}
    \item Player: No transitive dependencies exist.
    \item NPC: No transitive dependencies exist.
    \item Guild: No transitive dependencies exist.
    \item Quest: No transitive dependencies exist.
    \item Achievement: No transitive dependencies exist.
    \item Dimension: No transitive dependencies exist.
    \item Item: No transitive dependencies exist.
    \item Player\_Item: No transitive dependencies exist.
\end{itemize}

\end{document}
