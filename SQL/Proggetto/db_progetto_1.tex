\documentclass{article}
\usepackage{listings}
\usepackage{xcolor}
\usepackage{hyperref}
\usepackage[utf8]{inputenc}
\usepackage{longtable}

\lstset{
    basicstyle=\ttfamily\footnotesize,
    numbers=left,
    numberstyle=\tiny\color{gray},
    keywordstyle=\color{blue},
    commentstyle=\color{green},
    stringstyle=\color{red},
    breaklines=true,
    showstringspaces=false,
    frame=single,
    captionpos=b
}

\title{SQL Database Design and Queries for Dimensional Transfer Game}
\author{}

\begin{document}

\maketitle

\section{Project Description}
The project involves the creation of a database schema for a game. The database will track various entities such as players, non-player characters (NPCs), quests, items, achievements, guilds, and dimensions. It will also record relationships and actions such as quest completion, item ownership, and guild membership.

\section{SQL Operations}

% Player Queries Start Here
\subsection{View Player Stats}
This procedure displays the basic stats of a specific player, including their name, level, experience, and player score.
\begin{lstlisting}[language=sql,caption=View Player Stats]
DELIMITER //
CREATE PROCEDURE ViewPlayerStats(IN playerID INT)
BEGIN
    SELECT name, level, experience, player_score
    FROM Player
    WHERE player_id = playerID;
END //
DELIMITER ;
\end{lstlisting}

\subsection{View Inventory}
This procedure lists all items in a player's inventory, showing item name, quantity, type, and value.
\begin{lstlisting}[language=sql,caption=View Inventory]
DELIMITER //
CREATE PROCEDURE ViewInventory(IN playerID INT)
BEGIN
    SELECT Item.name AS ItemName, Player_Item.quantity, Item.type, Item.value
    FROM Player_Item
    INNER JOIN Item ON Player_Item.item_id = Item.item_id
    WHERE Player_Item.player_id = playerID;
END //
DELIMITER ;
\end{lstlisting}

\subsection{View Completed Quests}
This procedure shows the completed quests for a specific player by displaying the names of quests that the player has completed.
\begin{lstlisting}[language=sql,caption=View Completed Quests]
DELIMITER //
CREATE PROCEDURE ViewCompletedQuests(IN playerID INT)
BEGIN
    SELECT Quest.name AS QuestName
    FROM Completion
    INNER JOIN Quest ON Completion.quest_id = Quest.quest_id
    WHERE Completion.player_id = playerID AND Completion.state = TRUE;
END //
DELIMITER ;
\end{lstlisting}

\subsection{View Current Quests}
This procedure shows the current quests for a specific player by displaying the names of quests that the player is currently on.
\begin{lstlisting}[language=sql,caption=View Current Quests]
DELIMITER //
CREATE PROCEDURE ViewCurrentQuests(IN playerID INT)
BEGIN
    SELECT Quest.name AS QuestName
    FROM Player
    INNER JOIN Quest ON Player.quest_id = Quest.quest_id
    WHERE Player.player_id = playerID;
END //
DELIMITER ;
\end{lstlisting}

\subsection{View Achievements}
This procedure lists all achievements of a player, showing the achievement name and whether the player has earned it.
\begin{lstlisting}[language=sql,caption=View Achievements]
DELIMITER //
CREATE PROCEDURE ViewAchievements(IN playerID INT)
BEGIN
    SELECT Achievement.name AS AchievementName, Earn.state
    FROM Earn
    INNER JOIN Achievement ON Earn.achievement_id = Achievement.achievement_id
    WHERE Earn.player_id = playerID;
END //
DELIMITER ;
\end{lstlisting}

\subsection{View Guild Information}
This procedure displays guild information for a specific player, including the guild name, alignment, and guild leader.
\begin{lstlisting}[language=sql,caption=View Guild Information]
DELIMITER //
CREATE PROCEDURE ViewGuildInfo(IN playerID INT)
BEGIN
    SELECT Guild.name AS GuildName, Guild.alignment, Guild.guild_leader
    FROM Player
    INNER JOIN Guild ON Player.guild_id = Guild.guild_id
    WHERE Player.player_id = playerID;
END //
DELIMITER ;
\end{lstlisting}

% Developer/Moderator Queries Start Here
\subsection{Grant Permissions to a User}
This procedure grants specified permissions to a user on the database, allowing them to perform SELECT, INSERT, UPDATE, and DELETE operations.
\begin{lstlisting}[language=sql,caption=Grant Permissions to a User]
DELIMITER //
CREATE PROCEDURE GrantPermissions(IN username VARCHAR(255), IN hostname VARCHAR(255))
BEGIN
    SET @query = CONCAT('GRANT SELECT, INSERT, UPDATE, DELETE ON Dimensional_Transfer.* TO ?@?');
    SET @user = username;
    SET @host = hostname;
    PREPARE stmt FROM @query;
    EXECUTE stmt USING @user, @host;
    DEALLOCATE PREPARE stmt;
END //
DELIMITER ;
\end{lstlisting}

\subsection{Add Last Login Column}
This procedure adds a column to the Player table for tracking the last login time of players.
\begin{lstlisting}[language=sql,caption=Add Last Login Column]
DELIMITER //
CREATE PROCEDURE AddLastLoginColumn()
BEGIN
    ALTER TABLE Player
    ADD COLUMN last_login DATETIME;
END //
DELIMITER ;
\end{lstlisting}

\subsection{Update Last Login Time}
This procedure updates the last login time for a specific player to the current timestamp.
\begin{lstlisting}[language=sql,caption=Update Last Login Time]
DELIMITER //
CREATE PROCEDURE UpdateLastLogin(IN playerID INT)
BEGIN
    UPDATE Player
    SET last_login = NOW()
    WHERE player_id = playerID;
END //
DELIMITER ;
\end{lstlisting}

\subsection{Add a New Player}
This procedure adds a new player to the database, ensuring no duplicates, and checks the legality of items owned by the player.
\begin{lstlisting}[language=sql,caption=Add a New Player]
DELIMITER //
CREATE PROCEDURE AddPlayer(IN playerName VARCHAR(255), IN guildID INT, IN questID INT)
BEGIN
    DECLARE newPlayerID INT;

    IF NOT EXISTS (SELECT 1 FROM Player WHERE name = playerName) THEN
        INSERT INTO Player (name, guild_id, quest_id) VALUES (playerName, guildID, questID);
        SET newPlayerID = LAST_INSERT_ID();
        CALL CheckLegality(newPlayerID);
    END IF;
END //
DELIMITER ;
\end{lstlisting}

\subsection{Update Player Progression}
This procedure updates a player's experience and level, ensuring data consistency by using transactions.
\begin{lstlisting}[language=sql,caption=Update Player Progression]
DELIMITER //
CREATE PROCEDURE UpdatePlayerProgression(IN playerID INT, IN experienceGain INT)
BEGIN
    START TRANSACTION;
    UPDATE Player SET experience = experience + experienceGain WHERE player_id = playerID;
    UPDATE Player SET level = level + 1 WHERE player_id = playerID;
    CALL CheckLegality(playerID);
    COMMIT;
END //
DELIMITER ;
\end{lstlisting}

\subsection{Reset Player Progression}
This procedure resets a player's experience, level, and player score.
\begin{lstlisting}[language=sql,caption=Reset Player Progression]
DELIMITER //
CREATE PROCEDURE ResetPlayerProgression(IN playerID INT)
BEGIN
    UPDATE Player
    SET experience = 0, level = 1, player_score = 0
    WHERE player_id = playerID;
END //
DELIMITER ;
\end{lstlisting}

\subsection{Create Index on Player Name}
This procedure creates an index on the name column in the Player table to improve search performance.
\begin{lstlisting}[language=sql,caption=Create Index on Player Name]
DELIMITER //
CREATE PROCEDURE CreatePlayerNameIndex()
BEGIN
    CREATE INDEX idx_player_name ON Player(name);
END //
DELIMITER ;
\end{lstlisting}

\subsection{Get Players by Guild}
This procedure lists players belonging to a specific guild.
\begin{lstlisting}[language=sql,caption=Get Players by Guild]
DELIMITER //
CREATE PROCEDURE GetPlayersByGuild(IN guildID INT)
BEGIN
    SELECT Player.name AS PlayerName
    FROM Player
    WHERE Player.guild_id = guildID;
END //
DELIMITER ;
\end{lstlisting}

\subsection{Get Player Inventory Value}
This procedure calculates the total value of items in a player's inventory.
\begin{lstlisting}[language=sql,caption=Get Player Inventory Value]
DELIMITER //
CREATE PROCEDURE GetPlayerInventoryValue(IN playerID INT)
BEGIN
    SELECT SUM(Item.value * Player_Item.quantity) AS TotalValue
    FROM Player_Item
    INNER JOIN Item ON Player_Item.item_id = Item.item_id
    WHERE Player_Item.player_id = playerID;
END //
DELIMITER ;
\end{lstlisting}

\subsection{Get Players with Specific Achievement}
This procedure lists players who have earned a specific achievement.
\begin{lstlisting}[language=sql,caption=Get Players with Specific Achievement]
DELIMITER //
CREATE PROCEDURE GetPlayersWithAchievement(IN achievementName VARCHAR(255))
BEGIN
    SELECT Player.name AS PlayerName
    FROM Earn
    INNER JOIN Player ON Earn.player_id = Player.player_id
    INNER JOIN Achievement ON Earn.achievement_id = Achievement.achievement_id
    WHERE Achievement.name = achievementName AND Earn.state = TRUE;
END //
DELIMITER ;
\end{lstlisting}

\subsection{Check for Illegal Items}
This procedure lists players with illegal items.
\begin{lstlisting}[language=sql,caption=Check for Illegal Items]
DELIMITER //
CREATE PROCEDURE CheckForIllegalItems()
BEGIN
    SELECT Player.name AS PlayerName
    FROM Player
    WHERE player_id IN (
        SELECT player_id
        FROM Player_Item
        WHERE item_id IN (SELECT item_id FROM Item WHERE legality = FALSE)
    );
END //
DELIMITER ;
\end{lstlisting}

\subsection{Get Players with Illegal Items}
This procedure retrieves players with illegal items.
\begin{lstlisting}[language=sql,caption=Get Players with Illegal Items]
DELIMITER //
CREATE PROCEDURE GetPlayersWithIllegalItems()
BEGIN
    SELECT Player.name
    FROM Player
    WHERE player_id IN (
        SELECT player_id
        FROM Player_Item
        WHERE item_id IN (SELECT item_id FROM Item WHERE legality = FALSE)
    );
END //
DELIMITER ;
\end{lstlisting}

\subsection{Get Total Completed Quests by Players}
This procedure lists players with the total number of completed quests.
\begin{lstlisting}[language=sql,caption=Get Total Completed Quests by Players]
DELIMITER //
CREATE PROCEDURE GetTotalCompletedQuestsByPlayers()
BEGIN
    SELECT Player.name AS PlayerName, COUNT(Completion.quest_id) AS TotalCompletedQuests
    FROM Completion
    INNER JOIN Player ON Completion.player_id = Player.player_id
    WHERE Completion.state = TRUE
    GROUP BY Player.name
    ORDER BY TotalCompletedQuests DESC;
END //
DELIMITER ;
\end{lstlisting}

\subsection{Remove Illegal Items from Player Inventory}
This procedure removes illegal items from a player's inventory.
\begin{lstlisting}[language=sql,caption=Remove Illegal Items from Player Inventory]
DELIMITER //
CREATE PROCEDURE RemoveIllegalItemsFromInventory(IN playerID INT)
BEGIN
    DELETE FROM Player_Item
    WHERE player_id = playerID AND item_id IN (SELECT item_id FROM Item WHERE legality = FALSE);
END //
DELIMITER ;
\end{lstlisting}

\section{Schema Concettuale}

\begin{table}[h!]
\raggedright
\begin{tabular}{|l|l|l|}
\hline
\textbf{Entity 1} & \textbf{Relationship} & \textbf{Entity 2} \\ \hline
Player (1:N) & Belong & Guild (1:1) \\ \hline
Player (1:N) & Complete & Quest (1:N) \\ \hline
Player (1:N) & Own & Player\_Item (1:N) \\ \hline
Player\_Item (1:N) & Legal\_item & Item (1:N) \\ \hline
NPC (1:1) & Affiliation & Guild (1:1) \\ \hline
Dimension (1:1) & Complete & Quest (1:N) \\ \hline
\end{tabular}
\caption{Entity-Relationship Descriptions}
\end{table}

\section{Schema Logico}

\subsection{Entità}

\begin{longtable}{|l|l|l|l|l|}
\caption{Entities and their Attributes} \\
\hline
\textbf{Entità} & \textbf{Descrizione} & \textbf{Attributi 1} & \textbf{Attributi 2} & \textbf{Attributi 3} \\ \hline
\endfirsthead
\caption[]{Entities and their Attributes (continued)} \\
\hline
\textbf{Entità} & \textbf{Descrizione} & \textbf{Attributi 1} & \textbf{Attributi 2} & \textbf{Attributi 3} \\ \hline
\endhead
\hline \multicolumn{5}{|r|}{{Continued on next page}} \\ \hline
\endfoot
\hline
\endlastfoot
Player & A user of the game & \underline{player\_id} & name & level \\ \cline{3-5}
       & experience & \underline{[guild\_id]} & \underline{[quest\_id]} & \\ \cline{3-5}
       & player\_score & & & \\ \hline
NPC & Non-player character & \underline{npc\_id} & name & role \\ \cline{3-5}
    & alignment & \underline{[guild\_id]} & & \\ \hline
Quest & A task or mission & \underline{quest\_id} & name & description \\ \cline{3-5}
      & reward & quest\_status & & \\ \hline
Player\_Item & Items owned by a player & \underline{player\_item\_id} & \underline{[player\_id]} & \underline{[item\_id]} \\ \cline{3-5}
             & quantity & item\_condition & & \\ \hline
Item & Items in the game & \underline{item\_id} & name & type \\ \cline{3-5}
     & value & rarity & & \\ \hline
Achievement & Achievements earned by players & \underline{achievement\_id} & name & description \\ \cline{3-5}
            & achievement\_status & date\_earned & & \\ \hline
Guild & Groups that players can join & \underline{guild\_id} & name & alignment \\ \cline{3-5}
      & guild\_leader & guild\_points & & \\ \hline
Dimension & Different game worlds or levels & \underline{dimension\_id} & name & description \\ \cline{3-5}
          & difficulty\_level & & & \\ \hline
\end{longtable}

\subsection{Relazioni}

\begin{longtable}{|l|l|l|}
\caption{Relationships and their Attributes} \\
\hline
\textbf{Relazioni} & \textbf{Descrizione} & \textbf{Attributi 1} & \textbf{Attributi 2} & \textbf{Attributi 3} \\ \hline
\endfirsthead
\caption[]{Relationships and their Attributes (continued)} \\
\hline
\textbf{Relazioni} & \textbf{Descrizione} & \textbf{Attributi 1} & \textbf{Attributi 2} & \textbf{Attributi 3} \\ \hline
\endhead
\hline \multicolumn{4}{|r|}{{Continued on next page}} \\ \hline
\endfoot
\hline
\endlastfoot
Completion & Record of completed quests & \underline{[player\_id]} & \underline{[quest\_id]} & state \\ \hline
Belong     & Membership of players in guilds & \underline{[player\_id]} & \underline{[guild\_id]} & \underline{[guild\_name]} \\ \hline
Own        & Ownership of items by players & \underline{player\_item\_id} & \underline{[player\_id]} & \underline{[item\_id]} \\ \cline{3-5}
           & & quantity & item\_condition & \\ \hline
Affiliation & NPC affiliation with guilds & \underline{[npc\_id]} & \underline{[guild\_id]} & \underline{[guild\_name]} \\ \hline
\end{longtable}

\section{Redundancy Analysis}
Redundancy in the unnormalized schema can lead to data anomalies and inefficiencies. Detailed analysis of redundancy is as follows:
\begin{itemize}
    \item \textbf{Player}: Contains redundant attributes \underline{guild\_name} and \underline{quest\_name}. 
    \item \textbf{NPC}: Contains a redundant attribute \underline{guild\_name}.
    \item \textbf{Player\_Item}: Contains a redundant attribute \underline{item\_condition}. 
    \item \textbf{Achievement}: Contains a redundant attribute \underline{achievement\_status}.
    \item \textbf{Guild}: Contains redundant attributes \underline{guild\_leader} and \underline{guild\_points}.
\end{itemize}

\section{Restructuring with Analysis of Redundancy and Eventual Additions/Removals}
In this section, we remove redundancy from the schema. A derived attribute is one that can be calculated or inferred from other attributes in the database. We will remove such attributes and show the updated schema.

\subsection{Removal of Redundancy}
By removing the redundant attributes, the schema is optimized to avoid data anomalies and inefficiencies. Here is the detailed description of the changes made:

\begin{itemize}
    \item \textbf{Player}: The attributes \underline{guild\_name} and \underline{quest\_name} were removed. These attributes can be derived from \underline{guild\_id} and \underline{quest\_id}, respectively.
    \item \textbf{NPC}: The attribute \underline{guild\_name} was removed because it can be derived from \underline{guild\_id}.
    \item \textbf{Player\_Item}: The attribute \underline{item\_condition} was removed because it is a derived or calculated attribute.
    \item \textbf{Achievement}: The attribute \underline{achievement\_status} was removed because it can be inferred from \underline{date\_earned}.
    \item \textbf{Guild}: The attributes \underline{guild\_leader} and \underline{guild\_points} were removed because they may be derived or unnecessary depending on the use case.
\end{itemize}

\subsection{Output after Redundancy Removal}
The resulting entities and relationships after removing redundancy are:

\subsubsection{Entities}
\begin{itemize}
    \item \textbf{Player}: \texttt{player\_id, name, level, experience, guild\_id, quest\_id, player\_score}
    \item \textbf{NPC}: \texttt{npc\_id, name, role, alignment, guild\_id}
    \item \textbf{Quest}: \texttt{quest\_id, name, description, reward}
    \item \textbf{Player\_Item}: \texttt{player\_item\_id, player\_id, item\_id, quantity}
    \item \textbf{Item}: \texttt{item\_id, name, type, value}
    \item \textbf{Achievement}: \texttt{achievement\_id, name, description, date\_earned}
    \item \textbf{Guild}: \texttt{guild\_id, name, alignment}
    \item \textbf{Dimension}: \texttt{dimension\_id, name, description, difficulty\_level}
\end{itemize}

\subsubsection{Relationships}
\begin{itemize}
    \item \textbf{Completion}: \texttt{player\_id, quest\_id, state}
    \item \textbf{Belong}: \texttt{player\_id, guild\_id}
    \item \textbf{Own}: \texttt{player\_item\_id, player\_id, item\_id, quantity}
    \item \textbf{Affiliation}: \texttt{npc\_id, guild\_id}
\end{itemize}

\end{document}