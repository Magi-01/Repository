\documentclass[a4paper,12pt]{article}
\usepackage{tikz}
\usetikzlibrary{positioning, shapes.geometric, arrows.meta}
\usepackage{tabularx} % For tables with text wrapping
\usepackage{booktabs} % For better tables
\usepackage{hyperref}
\usepackage[a4paper, left=2cm, right=4cm, top=2.5cm, bottom=2.5cm]{geometry}
\usepackage{ragged2e} % for \RaggedRight in tables
\usepackage{array} % for >{\raggedright\arraybackslash}p{} in tables
\usepackage{float}
\usepackage[utf8]{inputenc}
\usepackage[italian]{babel} % Added Italian language support
\usepackage{amsmath}


\title{Design and Normalization Database on SuperMarket Aisle Management}
\author{Fadhla Mohamed Mutua - SM3201434}
\date{\today}

\begin{document}
\maketitle

\begin{abstract}
Questo rapporto presenta il progetto concettuale e logico di un database per la gestione delle corsie di un supermercato. Il sistema gestisce supermercati, corsie, articoli, produttori, distanze e la registrazione degli errori relativi agli articoli. Vengono spiegate le operazioni SQL utilizzate (vedi codice allegato), i diagrammi Entità-Relazione (ER), la ridondanza e le fasi di normalizzazione fino alla Terza Forma Normale (3NF).
\end{abstract}

\newpage
\tableofcontents
\newpage

\section{Introduzione}
Questo rapporto riguarda un sistema di database che supporta supermercati e produttori registrando gli errori generati da trigger automatici.

Assunzione: più supermercati possono avere una corsia con lo stesso nome, pertanto usando \texttt{AisleID} si garantisce che indipendentemente dal nome della corsia, l'\texttt{AisleID} corrisponda a uno specifico supermercato.

Nota: La relazione tra \texttt{Aisle} e \texttt{SuperMarket} nel PDF è chiamata \textit{Has Aisle} mentre nel codice allegato è denominata \texttt{AisleSuperMarket}.

\section*{2. Panoramica delle operazioni SQL}

Questa sezione delinea tutte le operazioni SQL principali implementate nel sistema di database, spiegandone il ruolo e come supportano la logica del dominio per la gestione e la conformità delle corsie.

\subsection*{2.1 Creazione delle Tabelle}

\textbf{Operazione: Definizione e inizializzazione dello schema}

Il primo insieme di operazioni definisce la struttura relazionale e lo schema del database. Tabelle come \texttt{SuperMarket}, \texttt{Aisle}, \texttt{Item}, \texttt{Producer}, \texttt{Distance}, \texttt{Contain}, \texttt{Manufactured\_By}, \texttt{ErrorMessages} e \texttt{ItemLogErrors} sono create usando \texttt{CREATE TABLE}. Queste istruzioni specificano chiavi primarie e esterne, tipi di dati e vincoli, stabilendo la base per l’archiviazione dei dati.

\subsection*{2.2 Trigger}

\textbf{Trigger: \texttt{trg\_check\_Item\_Aisle\_count}}

Questo trigger \texttt{BEFORE INSERT} garantisce che un articolo non sia posizionato in più corsie all’interno dello stesso supermercato. In caso di violazione genera un’eccezione SQL. Ciò assicura un’associazione uno a uno tra articolo e corsia per supermercato.

\textbf{Trigger: \texttt{trg\_log\_item\_wrong\_aisle}}

Questo trigger \texttt{AFTER INSERT} verifica se l’articolo è stato collocato nella corsia corretta utilizzando una funzione. In caso di non conformità, registra la violazione tramite un ID errore, usando messaggi standardizzati e timestamp.

\textbf{Trigger: \texttt{ReturnItem}}

Monitora gli articoli inseriti per scadenza. Se un articolo è scaduto, controlla se la distanza dal produttore è entro una soglia o se l’articolo è non deperibile. Registra quindi se l’articolo deve essere restituito o scartato.

\subsection*{2.3 Viste}

\textbf{Vista: \texttt{FullItemDetails}} — Risultato completo con join che mostra ogni articolo, la sua corsia, il supermercato e il produttore.

\textbf{Vista: \texttt{ItemWithProducers}} — Visualizza articoli e informazioni associate ai produttori.

\textbf{Vista: \texttt{ProducerSuperMarketDistance}} — Mappa le distanze tra produttori e supermercati.

\textbf{Vista: \texttt{WhereToStore}} — Abbina i tipi di conservazione degli articoli con i nomi delle corsie per la validazione della collocazione.

\textbf{Vista: \texttt{ItemErrorDetails}} — Fornisce registri dettagliati degli errori sugli articoli, inclusi messaggi e flag di scarto.

\subsection*{2.4 Funzioni memorizzate}

\textbf{Funzione: \texttt{fn\_validate\_aisle\_compliance}}

Applica le regole di dominio per la collocazione nelle corsie. Restituisce un errore leggibile dall’uomo se non conforme, altrimenti \texttt{NULL}.

\textbf{Funzione: \texttt{fn\_insert\_into\_error\_message}}

Inserisce un nuovo messaggio nella tabella \texttt{ErrorMessages} se non esiste e restituisce l’\texttt{ErrorID}.

\textbf{Funzione: \texttt{fn\_suggest\_correct\_aisle}}

Suggerisce l’\texttt{AisleID} più appropriato per un articolo in uno specifico supermercato basandosi su categoria e conservazione.

\subsection*{2.5 Procedure memorizzate}

\textbf{Procedura: \texttt{pr\_insert\_item\_log}} — Inserisce registrazioni di errori sugli articoli con timestamp e informazioni sull’errore.

\textbf{Procedura: \texttt{AddItemToAisle}} — Aggiunge un articolo a una corsia, verificando che la corsia appartenga al supermercato specificato.

\textbf{Procedura: \texttt{RemoveItemFromSuperMarket}} — Elimina un articolo da tutte le corsie di un supermercato.

\textbf{Procedura: \texttt{LogItemError}} — Registra manualmente un errore sull’articolo usando messaggi e dettagli dell’articolo.

\textbf{Procedura: \texttt{CleanExpiredItems}} — Elimina articoli scaduti e deperibili dalla tabella \texttt{Contain}.

\textbf{Procedura: \texttt{CheckItemCompliance}} — Verifica la conformità e opzionalmente suggerisce la collocazione corretta.

\textbf{Procedura: \texttt{sp\_check\_item\_placement}} — Scorre gli articoli, valida la collocazione e registra le non conformità.

\textbf{Procedura: \texttt{sp\_expiration\_check}} — Registra problemi relativi alla scadenza includendo informazioni su produttore e stato di scarto.

\subsection*{2.6 Eventi schedulati}

\textbf{Evento: \texttt{ev\_daily\_item\_placement\_check}} — Trigger giornaliero per eseguire \texttt{sp\_check\_item\_placement}.

\textbf{Evento: \texttt{ev\_daily\_expiration\_and\_cleanup}} — Pulizia giornaliera degli articoli scaduti e registrazione dello stato di scadenza.

\textbf{Evento: \texttt{ev\_daily\_expiration\_process}} — Evento wrapper che gestisce l’automazione di scadenza e pulizia.

\subsection*{2.7 Query}

\textbf{Query: Analisi storica degli errori}

\begin{verbatim}
SELECT
    em.ErrorMessage,
    i.ItemName, i.ItemStorageType,
    a.AisleID, a.AisleName AS IncorrectAisle,
    le.LogTime,
    le.ToBeThrown
FROM ItemLogErrors le
JOIN Item i ON le.ItemID = i.ItemID
JOIN Aisle a ON le.AisleID = a.AisleID
LEFT JOIN ErrorMessages em ON le.ErrorID = em.ErrorID
ORDER BY le.LogTime DESC;
\end{verbatim}

Fornisce un registro completo delle violazioni di collocazione o scadenza degli articoli con informazioni contestuali.

\section{Schema concettuale}

\subsection{Tabella dello schema concettuale}

\begin{table}[H]
\centering
\begin{tabularx}{\textwidth}{@{} l >{\RaggedRight\arraybackslash}X @{}}
\toprule
\textbf{Entità e Relazioni} & \textbf{Cardinalità (Relazionale)} \\ \midrule
Producer — Distance — SuperMarket & 1:N : 1:N \\
SuperMarket — Has\_Aisle — Aisle & 1:N : 1:1 \\
Aisle — Contains — Item & 1:1 : 0:N \\
Producer — Manufactured\_By — Item & 1:N : 1:1 \\
ItemLogErrors — Logs\_Item — Item & 0:N : 0:1 \\
ItemLogErrors — Logs\_Aisle — Aisle & 0:N : 0:1 \\
ItemLogErrors — Logs\_ErrorMessage — ErrorMessage & 1:N : 1:1 \\
\bottomrule
\end{tabularx}
\caption{Relazioni dello schema concettuale con cardinalità precise}
\label{tab:conceptual-schema}
\end{table}

\begin{table}[H]
\centering
\begin{tabularx}{\textwidth}{@{} l >{\RaggedRight\arraybackslash}X @{}}
\toprule
\textbf{Relazione} & \textbf{Descrizione} \\ \midrule
Distance & Collega i produttori ai supermercati con informazioni sulla distanza, utile per calcolare la logistica di trasporto e della catena di approvvigionamento (in questo caso per decidere se restituire o scartare un articolo). \\ 
Has\_Aisle & Associa le corsie ai supermercati, indicando la struttura interna di ciascun negozio. \\ 
Contains & Rappresenta gli articoli conservati nelle corsie, collegando l’inventario alla posizione. \\ 
Manufactured\_By & Collega gli articoli ai loro produttori, permettendo la tracciabilità dell’origine del prodotto. \\ 
Logs\_Item & Registra gli errori relativi ad articoli specifici, facilitando il tracciamento e la verifica degli errori. \\ 
Logs\_Aisle & Registra gli errori associati a corsie specifiche, aiutando a individuare problemi di conservazione. \\ 
Logs\_ErrorMessage & Associa i registri di errori a messaggi di errore standardizzati, permettendo una segnalazione coerente degli errori. \\ 
\bottomrule
\end{tabularx}
\caption{Relazioni dello schema concettuale e loro descrizione}
\label{tab:relationship-descriptions}
\end{table}

\subsection{Diagramma Entità-Relazione}
\newpage

\begin{figure}[H]
\hspace*{-2cm} % sposta a sinistra di 2cm
\begin{tikzpicture}[
  entity/.style={rectangle, draw=black, thick, minimum width=2.5cm, minimum height=1.2cm, fill=blue!10, font=\sffamily},
  relationship/.style={diamond, draw=black, thick, aspect=2, minimum width=2.5cm, minimum height=1.2cm, fill=red!10, font=\sffamily},
  line/.style={thick},
  every label/.style={font=\footnotesize},
  node distance=2cm and 3cm
]

% Entità e relazioni posizionate relativamente
\node[entity] (P) {Producer};
\node[relationship, right=4cm of P] (D) {Distance};
\node[entity, right=4cm of D] (S) {SuperMarket};

\node[relationship, below=2.5cm of S] (H) {Has\_Aisle};
\node[entity, below=1.5cm of H] (A) {Aisle};

\node[relationship, left=3.5cm of A] (C) {Contains};
\node[entity, below=2cm of C] (I) {Item};

\node[relationship, left=3.75cm of I] (M) {Manufactured\_By};

\node[relationship, below=2cm of A] (IL_A) {Logs\_Aisle};
\node[entity, below=2cm of IL_A] (Ie) {ItemLogErrors};
\node[relationship, left=3.5cm of Ie] (IL_I) {Logs\_Item};
\node[relationship, below=2cm of Ie] (IL_E) {Logs\_ErrorMessage};
\node[entity, below=2cm of IL_E] (E) {ErrorMessage};

% Disegna gli archi con cardinalità
\draw[line] (P) -- node[above] {1:N} (D);
\draw[line] (D) -- node[above] {1:N} (S);

\draw[line] (S) -- node[right] {1:N} (H);
\draw[line] (H) -- node[right] {1:1} (A);

\draw[line] (A) -- node[above] {1:1} (C);
\draw[line] (C) -- node[left] {0:N} (I);

\draw[line] (P) -- node[left] {1:N} (M);
\draw[line] (M) -- node[below] {1:1} (I);

\draw[line] (Ie) -- node[above] {1:N} (IL_I);
\draw[line] (IL_I) -- node[left] {0:1} (I);

\draw[line] (Ie) -- node[right] {1:N} (IL_A);
\draw[line] (IL_A) -- node[left] {0:1} (A);

\draw[line] (Ie) -- node[right] {1:N} (IL_E);
\draw[line] (IL_E) -- node[right] {1:1} (E);

\end{tikzpicture}
\caption{Diagramma Entità-Relazione per la gestione delle corsie}
\label{fig:er-diagram}
\end{figure}

\newpage
\subsection{Analisi della ridondanza}

\begin{itemize}
    \item \textbf{ItemLogErrors} ha molteplici relazioni binarie con \texttt{Item}, \texttt{Aisle} e \texttt{ErrorMessage}.
    \item Potrebbe essere sostituito da una relazione ternaria che collega direttamente \texttt{ItemLogErrors} con \texttt{Item}, \texttt{Aisle} e \texttt{ErrorMessage}.
    \item Poiché \texttt{ItemLogErrors} è generato da trigger, mantenere relazioni separate facilita le interrogazioni.
    \item Non ci sono entità o relazioni ridondanti per \texttt{Producer}, \texttt{SuperMarket}, \texttt{Aisle}, \texttt{Item} o \texttt{Distance}.
\end{itemize}

\newpage
\section{Schema logico}

\begin{table}[H]
\centering
\begin{tabularx}{\textwidth}{@{} l @{\hspace{1.5cm}} >{\RaggedRight\arraybackslash}X @{\hspace{1.5cm}} p{2.5cm} @{\hspace{1cm}} p{2.5cm} @{}}
\toprule
\textbf{Tabella} & \textbf{Attributi} & \textbf{PK} & \textbf{FK} \\ 
\midrule
Producer & ProducerID, ProducerName, ProducerLocation & \underline{ProducerID} & - \\ \hline
SuperMarket & SuperMarketID, SuperMarketName, SuperMarketLocation & \underline{SuperMarketID} & - \\ \hline
Aisle & AisleID, AisleName & \underline{AisleID} & -\\ \hline
Item & ItemID, ItemName, ItemCategory, ItemStorageType, ItemPerishable, ItemExpirationDate & \underline{ItemID} & - \\ \hline
Manufactured\_By & ItemID, ProducerID & \underline{ItemID} & ProducerID \\ \hline
Distance & ProducerID, SuperMarketID, Distance & \underline{ProducerID}, \underline{SuperMarketID} & ProducerID, SuperMarketID \\ \hline
Contain & AisleID, ItemID & \underline{AisleID}, \underline{ItemID} & AisleID, ItemID \\ \hline
Has\_Aisle & AisleID,  SuperMarketID, & \underline{AisleID}, \underline{ SuperMarketID,} & AisleID,  SuperMarketID, \\ \hline
ItemLogErrors & ErrorLogID, ItemID, AisleID, ErrorID, LogTime, ToBeThrown & \underline{ErrorLogID} & ItemID, AisleID, ErrorID \\ \hline
ErrorMessage & ErrorID, ErrorMessage & \underline{ErrorID} & - \\
\bottomrule
\end{tabularx}
\caption{Tabelle dello schema logico con chiavi primarie sottolineate}
\end{table}

\newpage
\section{Normalizzazione}

\subsection{Prima forma normale (1NF)}
\begin{itemize}
    \item Tutti gli attributi sono atomici e indivisibili.
    \item Non esistono array ripetuti poiché \texttt{ErrorMessage} è una stringa.
\end{itemize}

\subsection{Seconda forma normale (2NF)}
\begin{itemize}
    \item Non esistono dipendenze parziali su chiavi composte.
\end{itemize}

\subsection{Terza forma normale (3NF)}
\begin{itemize}
    \item Non sono presenti dipendenze transitive.
    \item Tutti gli attributi non chiave dipendono unicamente dalla chiave primaria.
\end{itemize}

\begin{table}[H]
\centering
\begin{tabularx}{\textwidth}{@{} l >{\RaggedRight\arraybackslash}X @{}}
\toprule
\textbf{Entità e Relazioni} & \textbf{Cardinalità (può/(o) deve : quantità \quad : \quad può/(o) deve : quantità)} \\ \midrule
Producer — Distance — SuperMarket & 1:N : 1:N \\
SuperMarket — Has\_Aisle — Aisle & 1:N : 1:1 \\
Aisle — Contains — Item & 1:1 : 0:N \\
Producer — Manufactured\_By — Item & 1:N : 1:1 \\
ItemLogErrors — Logs\_Item — Item & 0:N : 0:1 \\
ItemLogErrors — Logs\_Aisle — Aisle & 0:N : 0:1 \\
ItemLogErrors — Logs\_ErrorMessage — ErrorMessage & 1:N : 1:1 \\
\bottomrule
\end{tabularx}
\caption{Schema concettuale normalizzato con cardinalità precise}
\end{table}

\begin{table}[H]
\centering
\begin{tabularx}{\textwidth}{@{} l >{\RaggedRight\arraybackslash}X @{\hspace{1cm}} p{2.2cm} @{\hspace{0.8cm}} p{2.2cm} @{}}
\toprule
\textbf{Tabella} & \textbf{Attributi} & \textbf{PK} & \textbf{FK} \\ 
\midrule
Producer & ProducerID, ProducerName, ProducerLocation & \underline{ProducerID} & - \\ \hline
SuperMarket & SuperMarketID, SuperMarketName, SuperMarketLocation & \underline{SuperMarketID} & - \\ \hline
Aisle & AisleID, AisleName & \underline{AisleID} & -\\ \hline
Item & ItemID, ItemName, ItemCategory, ItemStorageType, ItemPerishable, ItemExpirationDate & \underline{ItemID} & - \\ \hline
Manufactured\_By & ItemID, ProducerID & \underline{ItemID} & ProducerID \\ \hline
Distance & ProducerID, SuperMarketID, Distance & \underline{ProducerID}, \underline{SuperMarketID} & ProducerID, SuperMarketID \\ \hline
Contain & AisleID, ItemID & \underline{AisleID}, \underline{ItemID} & AisleID, ItemID \\ \hline
Has\_Aisle & AisleID,  SuperMarketID, & \underline{AisleID}, \underline{ SuperMarketID,} & AisleID,  SuperMarketID, \\ \hline
ItemLogErrors & ErrorLogID, ItemID, AisleID, ErrorID, LogTime, ToBeThrown & \underline{ErrorLogID} & ItemID, AisleID, ErrorID \\ \hline
ErrorMessage & ErrorID, ErrorMessage & \underline{ErrorID} & - \\
\bottomrule
\end{tabularx}
\caption{Schema logico normalizzato}
\end{table}

\section{Conclusione}

Il processo di normalizzazione conferma che lo schema aderisce alla 3NF, supportando il tracciamento automatico degli errori senza introdurre ridondanze.

\end{document}