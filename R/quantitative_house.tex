\documentclass{article}
\usepackage{listings}
\begin{document}
\begin{center}
\begin{lstlisting}[language=R]
House_prices = read.csv2("./house_price.csv", sep = ",", stringsAsFactors=T, row.names=1) attach(House_prices) #' --------------------- #' Analisi Univariata #' ---------------------

univariate_analysis <- function(data, colname){

variance = round(var(data), digits = 2) standard_deviation = round(sd(data), digits = 2) result = list()

# Add summary statistics to the list cat("\n") result[[colname]] = list( #' Informazioni contenenti il minimo, il 1° e il 3° quartile, #' la media (2° quartile), la mediana (valore centrale), #' il massimo e la quantità di dati non validi. Summary = summary(data), #' Informazioni contenenti la varianza Variance = format(variance, scientific = FALSE), #' Informazioni contenenti la deviazione standard Standard_Deviation = format(standard_deviation, scientific = FALSE), #' Informazioni contenenti la moda Moda = calculate_mode(data) ) cat("\n")

return(result) }

calculate_mode = function(data) { uniq_vals = unique(na.omit(data)) freq_vals = tabulate(match(na.omit(data), uniq_vals)) max_mode_vals = uniq_vals[freq_vals == max(freq_vals)] min_mode_vals = uniq_vals[freq_vals == min(freq_vals)]

if (length(max_mode_vals)>=5){ mode_info = list( Max_amount_Info = "Too many max Values", Minimum_amount_Info = list(c(min_Value = min_mode_vals, freq = min(freq_vals))) ) } else if (length(min_mode_vals)>=5){ mode_info = list( Max_amount_Info = list(c(max_Value = max_mode_vals, freq = max(freq_vals))), Minimum_amount_Info = "Too many min Values" ) } else if (length(min_mode_vals)<5 & length(max_mode_vals)<5){ mode_info = list( Max_amount_Info = list(c(max_Value = max_mode_vals, freq = max(freq_vals))), Minimum_amount_Info = list(c(min_Value = min_mode_vals, freq = min(freq_vals))) ) } else{ mode_info = "Too many mins and maxs" }

return(mode_info) }

sum_info_h=c() for (ele in colnames(House_prices[,c(-1,-2,-17,-18)])){ colummn = na.omit(House_prices[,ele]) if (is.numeric(colummn)){ sum_info_h = c(sum_info_h, univariate_analysis(colummn,ele)) } }

#' ----------------------------- #' 1: Visualizazione LotFrontage #' -----------------------------

visualization_LotFrontage <- function(data, sum_info_h){ par(mfrow = c(2,2)) print(sum_info_h) #' La presenza di molti outlier suggerisce che i dati #' non sono distribuiti normalmente e essendo maggiormente a sinistra #' allora sono positivamente asimmetrici. boxplot(data, horizontal = T, main = paste("Boxplot of LotFrontage")) ##' Istogramma con curva normale positivamente asimmetrici hist(data, probability = T, col = 2, main = paste("Hist of LotFrontage")) curve(dnorm(x, mean = mean(data), sd = sd(data)), add=T, col = 4) print(density(data)) ##' Distribuzione della densita di LotFrontage plot(density(data), main = "Desnity Distribution of LotFrontage") #' Si puo veder che qqnorm(data, col = 3, main = paste("QQnorm of LotFrontage")) qqline(data, col = 5, pch = 10) #' Possiamo dedurre che la distribuzione è per lo più positivamente asimetrica #' e che la maggior parte degli lotti ha un percorso verso la strada #' compreso tra 59 e 80 piedi, essendo rispettivamente il 1° e il 3° quantile e maggiormente #' alineati alla retta qq della norma. #' Rimangono ancora degli outlier

readline("Press enter to continue: ")

#' Rimozione degli outlier attraverso intervallo di confidenza 95% per avvicinare alla norma mean_data = mean(data) sd_data = sd(data) confidence_level = 0.95 z_score = qnorm((1 + confidence_level) / 2)

lower_bound = mean_data - z_score * sd_data upper_bound = mean_data + z_score * sd_data

data_in_ci = data[data >= lower_bound & data <= upper_bound] print(list(summary = summary(data_in_ci), Variance = var(data_in_ci), Standard_deviation = sd(data_in_ci))) boxplot(data_in_ci, horizontal = T, main = paste("Boxplot of LotFrontage-rm-anomaly")) hist(data_in_ci, probability = T, col = 2, main = paste("Hist of LotFrontage")) curve(dnorm(x, mean = mean(data_in_ci), sd = sd(data_in_ci)), add=T, col = 4) print(density(data_in_ci)) plot(density(data_in_ci), main = "Density Distribution of LotFrontage") qqnorm(data_in_ci, col = 3, main = paste("QQnorm of LotFrontage")) qqline(data_in_ci, col = 5, pch = 10) #' Possiamo deddure che la distribuzione segue per lo piu quella gaussiana #' e che la maggior parte degli lotti ha un percorso verso la strada #' compreso tra 60 e 80 piedi, essendo rispettivamente il 1° e il 3° quantile e con maggior #' densita alineati alla retta qq della norma. } visualization_LotFrontage(na.omit(House_prices[,"LotFrontage"]), sum_info_h$LotFrontage)

#' ----------------------------- #' 2: Visualizazione LotArea #' -----------------------------

visualization_LotArea <- function(data, sum_info_h){ par(mfrow = c(2,2)) print(sum_info_h) boxplot(data, horizontal = T, main = paste("Boxplot of LotArea")) ##' Parametro di lisciamento h = 0.9(min(sd(data), IQR(data))/1.34)NROW(data)^(-1/5) ##' numeri di intervalli per l'istogramma br = (max(data)-min(data))/h ##' Istogramma con curva normale positivamente asimmetrici ##' con intervallini di grandezza br print(paste("br = ",br)) hist(data, probability = T, col = 2, main = paste("Hist of LotArea"), breaks = br) curve(dnorm(x, mean = mean(data), sd = sd(data)), add=T, col = 4) print(density(data)) ##' Distribuzione della densita di LotArea plot(density(data, bw=h), main = "Density Distribution of LotArea") qqnorm(data, col = 3, main = paste("QQnorm of LotArea")) qqline(data, col = 5, pch = 10) #' Possiamo dedurre che la distribuzione è positivamente asimetrica #' e che la maggior parte degli lotti ha un area compreso #' tra 7554 e 11602 piedi^2, essendo rispettivamente il 1° e il 3° quantile e #' e con maggior densita alineati alla retta qq della norma. #' Rimangono ancora degli outlier

readline("Press enter to continue: ")

#' Rimozione degli outlier attraverso intervallo di confidenza 95% per avvicinare alla norma mean_data = mean(data) sd_data = sd(data) confidence_level = 0.95 z_score = qnorm((1 + confidence_level) / 2)

lower_bound = mean_data - z_score * sd_data upper_bound = mean_data + z_score * sd_data

data_in_ci = data[data >= lower_bound & data <= upper_bound] print(list(summary = summary(data_in_ci), Variance = var(data_in_ci), Standard_deviation = sd(data_in_ci))) boxplot(data_in_ci, horizontal = T, main = paste("Boxplot of LotArea-rm-anomaly")) h = 0.9(min(sd(data_in_ci), IQR(data_in_ci))/1.34)NROW(data_in_ci)^(-1/5) br = (max(data_in_ci)-min(data_in_ci))/h print(paste("br_CI = ",br)) hist(data_in_ci, probability = T, col = 2, main = paste("Hist of LotArea"), breaks = br) curve(dnorm(x, mean = mean(data_in_ci), sd = sd(data_in_ci)), add=T, col = 4) print(density(data_in_ci)) plot(density(data_in_ci, bw=h), main = "Density Distribution of LotArea") qqnorm(data_in_ci, col = 3, main = paste("QQnorm of LotArea")) qqline(data_in_ci, col = 5, pch = 10) #' Possiamo dedurre che la distribuzione è positivamente asimetrica #' e che la maggior parte degli lotti ha un area compreso #' tra 7500 e 11362 piedi^2, essendo rispettivamente il 1° e il 3° quantile e con maggior #' densita alineati alla retta qq della norma. } visualization_LotArea(na.omit(House_prices[,"LotArea"]), sum_info_h$LotArea)

#' ----------------------------- #' 3: Visualizazione YearBuilt #' -----------------------------

visualization_YearBuilt <- function(data, sum_info_h){ par(mfrow = c(2,2)) print(sum_info_h) boxplot(data, horizontal = T, main = paste("Boxplot of YearBuilt")) h = 0.9(min(sd(data), IQR(data))/1.34)NROW(data)^(-1/5) br = as.integer((max(data)-min(data))/h) ##' Istogramma con curva normale positivamente asimmetrici ##' con intervallini di grandezza br print(paste("br = ",br)) hist(data, probability = T, col = 2, main = paste("Hist of YearBuilt"), breaks = br) print(density(data, bw=h)) ##' Distribuzione della densita di YearBuilt plot(density(data, bw=h), main = "Density Distribution of YearBuilt") #' Possiamo dedurre che la distribuzione è negativamente asimetrica #' e quindi il numero di edifici costruiti crescono ogni anno #' La maggior parte degli edifici erano costruiti fra 1954 e 2000, #' essendo rispettivamente il 1° e il 3° quantile e l'anno con #' è il massimo costruitto è nel 2000 con 67 edifici } visualization_YearBuilt(na.omit(House_prices[,"YearBuilt"]), sum_info_h$YearBuilt)

#' ----------------------------- #' 4: Visualizazione YearRemodAd #' -----------------------------

visualization_YearRemodAdd <- function(data, sum_info_h){ par(mfrow = c(2,2)) print(sum_info_h) boxplot(data, horizontal = T, main = paste("Boxplot of YearRemodAdd")) h = 0.9(min(sd(data), IQR(data))/1.34)NROW(data)^(-1/5) br = as.integer((max(data)-min(data))/h) print(paste("br = ",br)) hist(data, probability = T, col = 2, main = paste("Hist of YearRemodAdd"), breaks = br) print(density(data)) plot(density(data, bw=h), main = "Desnity Distribution of YearRemodAdd") qqnorm(data, col = 3, main = paste("QQnorm of YearRemodAdd")) qqline(data, col = 5, pch = 10) #' Possiamo notare che la distribuzione ha un pico negli anni 50 con il massimo di rimodellazione #' nel 1950 con 178 edifici. Dopodichè, si ha un calo quasi immediato #' e i numeri di edifici rimodellati resta quasi costante fino agli anni 80 #' dove fra il 1980 al 1985 si nota il minimo numero di rimodellazione. Dopodiche, #' la distribuzione inizia a crescere quasi costantemente con pico negli anni fra 2000 e 2005 } visualization_YearRemodAdd(na.omit(House_prices[,"YearRemodAdd"]), sum_info_h$YearRemodAdd)

#' ----------------------------- #' 5: Visualizazione MasVnrArea #' -----------------------------

visualization_MasVnrArea <- function(data, sum_info_h){ par(mfrow = c(2,2)) print(sum_info_h) boxplot(data, horizontal = T, main = paste("Boxplot of MasVnrArea")) h = 0.9(min(sd(data), IQR(data))/1.34)NROW(data)^(-1/5) br = (max(data)-min(data))/h hist(data, probability = T, col = 2, main = paste("Hist of MasVnrArea"), breaks = br) curve(dnorm(x, mean = mean(data), sd = sd(data)), add=T, col = 4) print(density(data)) plot(density(data, bw=h), main = "Density Distribution of MasVnrArea") qqnorm(data, col = 3, main = paste("QQnorm of MasVnrArea")) qqline(data, col = 5, pch = 10) #' Possiamo dedurre che la distribuzione è positivamente asimetrica. #' Si puo osservare che l'area di MasVnrArea è prelevantemete 0 qundi non applicato. #' Per fare un analisi sugli edifici che hanno il Masonry veneer, serve ignorare quelli senza

readline("Press enter to continue: ")

#' Rimozione dello 0 imp_data = data imp_data[imp_data == 0] = NA imp_data = na.omit(imp_data) print(list(summary = summary(imp_data), Variance = var(imp_data), Standard_deviation = sd(imp_data))) boxplot(imp_data, horizontal = T, main = paste("Boxplot of MasVnrArea")) h = 0.9(min(sd(imp_data), IQR(imp_data))/1.34)NROW(imp_data)^(-1/5) br = (max(imp_data)-min(imp_data))/h hist(imp_data, probability = T, col = 2, main = paste("Hist of MasVnrArea"), breaks = br) curve(dnorm(x, mean = mean(imp_data), sd = sd(imp_data)), add=T, col = 4) print(density(imp_data)) plot(density(imp_data, bw=h), main = "Density Distribution of MasVnrArea") qqnorm(imp_data, col = 3, main = paste("QQnorm of MasVnrArea")) qqline(imp_data, col = 5, pch = 10) #' Possiamo dedurre che la distribuzione è positivamente asimetrica #' e che la maggior parte degli edifici ha un area coperto #' da Masonry veneer tra 113.0 e 330.5 piedi^2, essendo rispettivamente il 1° e il 3° quantile #' e con maggior densita alineati alla retta qq della norma. #' Rimangono ancora degli outlier

readline("Press enter to continue: ")

#' Rimozione degli outlier attraverso intervallo di confidenza 95% per avvicinare alla norma mean_data = mean(imp_data) sd_data = sd(imp_data) confidence_level = 0.95 z_score = qnorm((1 + confidence_level) / 2)

lower_bound = mean_data - z_score * sd_data upper_bound = mean_data + z_score * sd_data

data_in_ci = imp_data[imp_data >= lower_bound & imp_data <= upper_bound] print(list(summary = summary(data_in_ci), Variance = var(data_in_ci), Standard_deviation = sd(data_in_ci))) boxplot(data_in_ci, horizontal = T, main = paste("Boxplot of MasVnrArea")) h = 0.9(min(sd(data_in_ci), IQR(data_in_ci))/1.34)NROW(data_in_ci)^(-1/5) br = (max(data_in_ci)-min(data_in_ci))/h print(paste("br_CI = ",br)) hist(data_in_ci, probability = T, col = 2, main = paste("Hist of MasVnrArea"), breaks = br) curve(dnorm(x, mean = mean(data_in_ci), sd = sd(data_in_ci)), add=T, col = 4) print(density(data_in_ci)) plot(density(data_in_ci, bw=h), main = "Density Distribution of MasVnrArea") qqnorm(data_in_ci, col = 3, main = paste("QQnorm of MasVnrArea")) qqline(data_in_ci, col = 5, pch = 10) #' Possiamo dedurre nella distribuzione che la maggior parte degli edifici ha un area coperto #' da Masonry veneer tra 108.5 e 304.0 piedi^2, essendo rispettivamente il 1° e il 3° quantile #' e con maggior densita alineati alla retta qq della norma. } visualization_MasVnrArea(na.omit(House_prices[,"MasVnrArea"]), sum_info_h$MasVnrArea)

#' ----------------------------- #' 6: Visualizazione BsmtFinSF1 #' -----------------------------

visualization_BsmtFinSF1 <- function(data, sum_info_h){ par(mfrow = c(2,2)) print(sum_info_h) boxplot(data, horizontal = TRUE, main = "Boxplot of BsmtFinSF1") h = 0.9 * (min(sd(data), IQR(data)) / 1.34) * NROW(data)^(-1/5) br = (max(data) - min(data)) / h hist(data, probability = TRUE, col = 2, main = "Hist of BsmtFinSF1", breaks = br) curve(dnorm(x, mean = mean(data), sd = sd(data)), add = TRUE, col = 4) print(density(data)) plot(density(data, bw =h), main = "Density Distribution of BsmtFinSF1") qqnorm(data, col = 3, main = "QQnorm of BsmtFinSF1") qqline(data, col = 5, pch = 10) #' Possiamo dedurre che la distribuzione è positivamente asimetrica. #' Si puo osservare che l'area finito di semiterrato finito di tipo 1 è prelevantemete 0 #' qundi non costruitto. #' Per fare un analisi sugli edifici che hanno il semiterrato finito di tipo 1, #' serve ignorare quelli senza

readline("Press enter to continue: ")

#' Rimozione dello 0 imp_data = data imp_data[imp_data == 0] = NA imp_data = na.omit(imp_data) print(list(summary = summary(imp_data), Variance = var(imp_data), Standard_deviation = sd(imp_data))) boxplot(imp_data, horizontal = T, main = paste("Boxplot of BsmtFinSF1")) h = 0.9(min(sd(imp_data), IQR(imp_data))/1.34)NROW(imp_data)^(-1/5) br = (max(imp_data)-min(imp_data))/h hist(imp_data, probability = T, col = 2, main = paste("Hist of BsmtFinSF1"), breaks = br) curve(dnorm(x, mean = mean(imp_data), sd = sd(imp_data)), add=T, col = 4) print(density(imp_data)) plot(density(imp_data, bw=h), main = "Density Distribution of BsmtFinSF1") qqnorm(imp_data, col = 3, main = paste("QQnorm of BsmtFinSF1")) qqline(imp_data, col = 5, pch = 10) #' Possiamo dedurre che la distribuzione è positivamente asimetrica #' e che la maggior parte degli edifici ha un semiterrato finito di tipo 1 di area compresa #' tra 371.0 e 867.0 piedi^2, essendo rispettivamente il 1° e il 3° quantile #' e con maggior densita alineati alla retta qq della norma. #' Rimangono ancora degli outlier

readline("Press enter to continue: ")

#' Rimozione degli outlier attraverso intervallo di confidenza 95% per avvicinare alla norma mean_data = mean(imp_data) sd_data = sd(imp_data) confidence_level = 0.95 z_score = qnorm((1 + confidence_level) / 2)

lower_bound = mean_data - z_score * sd_data upper_bound = mean_data + z_score * sd_data

data_in_ci = imp_data[imp_data >= lower_bound & imp_data <= upper_bound] print(list(summary = summary(data_in_ci), Variance = var(data_in_ci), Standard_deviation = sd(data_in_ci))) boxplot(data_in_ci, horizontal = T, main = paste("Boxplot of BsmtFinSF1")) h = 0.9(min(sd(data_in_ci), IQR(data_in_ci))/1.34)NROW(data_in_ci)^(-1/5) br = (max(data_in_ci)-min(data_in_ci))/h print(paste("br_CI = ",br)) hist(data_in_ci, probability = T, col = 2, main = paste("Hist of BsmtFinSF1"), breaks = br) curve(dnorm(x, mean = mean(data_in_ci), sd = sd(data_in_ci)), add=T, col = 4) print(density(data_in_ci)) plot(density(data_in_ci, bw=h), main = "Density Distribution of BsmtFinSF1") qqnorm(data_in_ci, col = 3, main = paste("QQnorm of BsmtFinSF1")) qqline(data_in_ci, col = 5, pch = 10) #' Possiamo dedurre nella distribuzione semi-normato che la maggior parte degli edifici #' ha un semiterrato finito di tipo 1 di area compresa tra 364.8 e 833.8 piedi^2, #' essendo rispettivamente il 1° e il 3° quantile #' e con maggior densita alineati alla retta qq della norma. } visualization_BsmtFinSF1(na.omit(House_prices[,"BsmtFinSF1"]), sum_info_h$BsmtFinSF1)

#' ----------------------------- #' 7: Visualizazione BsmtFinSF2 #' -----------------------------

visualization_BsmtFinSF2 <- function(data, sum_info_h){ par(mfrow = c(2,2)) print(sum_info_h) boxplot(data, horizontal = TRUE, main = "Boxplot of BsmtFinSF2") h = 0.9 * (min(sd(data), IQR(data)) / 1.34) * NROW(data)^(-1/5) br = (max(data) - min(data)) / h hist(data, probability = TRUE, col = 2, main = "Hist of BsmtFinSF2") curve(dnorm(x, mean = mean(data), sd = sd(data)), add = TRUE, col = 4) print(density(data)) plot(density(data), main = "Density Distribution of BsmtFinSF2") qqnorm(data, col = 3, main = "QQnorm of BsmtFinSF2") qqline(data, col = 5, pch = 10) #' Possiamo dedurre che la distribuzione è positivamente asimetrica. #' Si puo osservare che l'area finito di semiterrato finito di tipo 2 è prelevantemete 0 #' qundi non costruitto. #' Per fare un analisi sugli edifici che hanno il semiterrato finito di tipo 2, #' serve ignorare quelli senza

readline("Press enter to continue: ")

#' Rimozione dello 0 imp_data = data imp_data[imp_data == 0] = NA imp_data = na.omit(imp_data) print(list(summary = summary(imp_data), Variance = var(imp_data), Standard_deviation = sd(imp_data))) boxplot(imp_data, horizontal = T, main = paste("Boxplot of BsmtFinSF2")) h = 0.9(min(sd(imp_data), IQR(imp_data))/1.34)NROW(imp_data)^(-1/5) br = (max(imp_data)-min(imp_data))/h hist(imp_data, probability = T, col = 2, main = paste("Hist of BsmtFinSF2"), breaks = br) curve(dnorm(x, mean = mean(imp_data), sd = sd(imp_data)), add=T, col = 4) print(density(imp_data)) plot(density(imp_data, bw=h), main = "Density Distribution of BsmtFinSF2") qqnorm(imp_data, col = 3, main = paste("QQnorm of BsmtFinSF2")) qqline(imp_data, col = 5, pch = 10) #' Possiamo dedurre che la distribuzione è positivamente asimetrica #' e che la maggior parte degli edifici ha un semiterrato finito di tipo 2 di area compresa #' tra 178.5 e 551.0 piedi^2, essendo rispettivamente il 1° e il 3° quantile #' e con maggior densita alineati alla retta qq della norma. #' Rimangono ancora degli outlier

readline("Press enter to continue: ")

#' Rimozione degli outlier attraverso intervallo di confidenza 95% per avvicinare alla norma mean_data = mean(imp_data) sd_data = sd(imp_data) confidence_level = 0.95 z_score = qnorm((1 + confidence_level) / 2)

lower_bound = mean_data - z_score * sd_data upper_bound = mean_data + z_score * sd_data

data_in_ci = imp_data[imp_data >= lower_bound & imp_data <= upper_bound] print(list(summary = summary(data_in_ci), Variance = var(data_in_ci), Standard_deviation = sd(data_in_ci))) boxplot(data_in_ci, horizontal = T, main = paste("Boxplot of BsmtFinSF2")) h = 0.9(min(sd(data_in_ci), IQR(data_in_ci))/1.34)NROW(data_in_ci)^(-1/5) br = (max(data_in_ci)-min(data_in_ci))/h print(paste("br_CI = ",br)) hist(data_in_ci, probability = T, col = 2, main = paste("Hist of BsmtFinSF2"), breaks = br) curve(dnorm(x, mean = mean(data_in_ci), sd = sd(data_in_ci)), add=T, col = 4) print(density(data_in_ci)) plot(density(data_in_ci, bw=h), main = "Density Distribution of BsmtFinSF2") qqnorm(data_in_ci, col = 3, main = paste("QQnorm of BsmtFinSF2")) qqline(data_in_ci, col = 5, pch = 10) #' Possiamo dedurre nella distribuzione semi-normato che la maggior parte degli edifici #' ha un semiterrato finito di tipo 2 di area compresa tra 173.8 e 512.2 piedi^2, #' essendo rispettivamente il 1° e il 3° quantile #' e con maggior densita alineati alla retta qq della norma. } visualization_BsmtFinSF2(na.omit(House_prices[,"BsmtFinSF2"]), sum_info_h$BsmtFinSF2)

#' ----------------------------- #' 8: Visualizazione BsmtUnfSF #' -----------------------------

visualization_BsmtUnfSF <- function(data, sum_info_h){ par(mfrow = c(2,2)) print(sum_info_h) boxplot(data, horizontal = TRUE, main = "Boxplot of BsmtUnfSF") h = 0.9 * (min(sd(data), IQR(data)) / 1.34) * NROW(data)^(-1/5) br = (max(data) - min(data)) / h hist(data, probability = TRUE, col = 2, main = "Hist of BsmtUnfSF") curve(dnorm(x, mean = mean(data), sd = sd(data)), add = TRUE, col = 4) print(density(data)) plot(density(data), main = "Density Distribution of BsmtUnfSF") qqnorm(data, col = 3, main = "QQnorm of BsmtUnfSF") qqline(data, col = 5, pch = 10) #' Possiamo dedurre che la distribuzione è positivamente asimetrica. #' Si puo osservare che l'area di semiterrato non finito è prelevantemete 0 #' qundi non costruitto. #' Per fare un analisi sugli edifici che hanno il semiterrato non finito, #' serve ignorare quelli con

readline("Press enter to continue: ")

#' Rimozione dello 0 imp_data = data imp_data[imp_data == 0] = NA imp_data = na.omit(imp_data) print(list(summary = summary(imp_data), Variance = var(imp_data), Standard_deviation = sd(imp_data))) boxplot(imp_data, horizontal = T, main = paste("Boxplot of BsmtUnfSF")) h = 0.9(min(sd(imp_data), IQR(imp_data))/1.34)NROW(imp_data)^(-1/5) br = (max(imp_data)-min(imp_data))/h hist(imp_data, probability = T, col = 2, main = paste("Hist of BsmtUnfSF"), breaks = br) curve(dnorm(x, mean = mean(imp_data), sd = sd(imp_data)), add=T, col = 4) print(density(imp_data)) plot(density(imp_data, bw=h), main = "Density Distribution of BsmtUnfSF") qqnorm(imp_data, col = 3, main = paste("QQnorm of BsmtUnfSF")) qqline(imp_data, col = 5, pch = 10) #' Possiamo dedurre che la distribuzione è positivamente asimetrica #' e che la maggior parte degli edifici ha un semiterrato non finito di area compresa #' tra 288.0 e 843.2 piedi^2, essendo rispettivamente il 1° e il 3° quantile #' e con maggior densita alineati alla retta qq della norma. #' Rimangono ancora degli outlier

readline("Press enter to continue: ")

#' Rimozione degli outlier attraverso intervallo di confidenza 95% per avvicinare alla norma mean_data = mean(imp_data) sd_data = sd(imp_data) confidence_level = 0.95 z_score = qnorm((1 + confidence_level) / 2)

lower_bound = mean_data - z_score * sd_data upper_bound = mean_data + z_score * sd_data

data_in_ci = imp_data[imp_data >= lower_bound & imp_data <= upper_bound] print(list(summary = summary(data_in_ci), Variance = var(data_in_ci), Standard_deviation = sd(data_in_ci))) boxplot(data_in_ci, horizontal = T, main = paste("Boxplot of BsmtUnfSF")) h = 0.9(min(sd(data_in_ci), IQR(data_in_ci))/1.34)NROW(data_in_ci)^(-1/5) br = (max(data_in_ci)-min(data_in_ci))/h print(paste("br_CI = ",br)) hist(data_in_ci, probability = T, col = 2, main = paste("Hist of BsmtUnfSF"), breaks = br) curve(dnorm(x, mean = mean(data_in_ci), sd = sd(data_in_ci)), add=T, col = 4) print(density(data_in_ci)) plot(density(data_in_ci, bw=h), main = "Density Distribution of BsmtUnfSF") qqnorm(data_in_ci, col = 3, main = paste("QQnorm of BsmtUnfSF")) qqline(data_in_ci, col = 5, pch = 10) #' Possiamo dedurre nella distribuzione semi-normato che la maggior parte degli edifici #' ha un semiterrato non finito di area compresa tra 278.5 e 780.0 piedi^2, #' essendo rispettivamente il 1° e il 3° quantile #' e con maggior densita alineati alla retta qq della norma. } visualization_BsmtUnfSF(na.omit(House_prices[,"BsmtUnfSF"]), sum_info_h$BsmtUnfSF)

#' ----------------------------- #' 9: Visualizazione TotalBsmtSF #' -----------------------------

visualization_TotalBsmtSF <- function(data, sum_info_h){ par(mfrow = c(2,2)) print(sum_info_h) boxplot(data, horizontal = TRUE, main = "Boxplot of TotalBsmtSF") h = 0.9 * (min(sd(data), IQR(data)) / 1.34) * NROW(data)^(-1/5) br = (max(data) - min(data)) / h hist(data, probability = TRUE, col = 2, main = "Hist of TotalBsmtSF", breaks = br) curve(dnorm(x, mean = mean(data), sd = sd(data)), add = TRUE, col = 4) print(density(data)) plot(density(data, bw=h), main = "Density Distribution of TotalBsmtSF") qqnorm(data, col = 3, main = "QQnorm of TotalBsmtSF") qqline(data, col = 5, pch = 10) #' Possiamo dedurre che la distribuzione è positivamente asimetrica. #' Si puo osservare che l'area totale semiterrato è prelevantemete 0 #' qundi non costruitto. #' Per fare un analisi sugli edifici che hanno il semiterrato non finito, #' serve ignorare quelli senza

readline("Press enter to continue: ")

#' Rimozione dello 0 imp_data = data imp_data[imp_data == 0] = NA imp_data = na.omit(imp_data) print(list(summary = summary(imp_data), Variance = var(imp_data), Standard_deviation = sd(imp_data))) boxplot(imp_data, horizontal = T, main = paste("Boxplot of BsmtUnfSF")) h = 0.9(min(sd(imp_data), IQR(imp_data))/1.34)NROW(imp_data)^(-1/5) br = (max(imp_data)-min(imp_data))/h hist(imp_data, probability = T, col = 2, main = paste("Hist of BsmtUnfSF"), breaks = br) curve(dnorm(x, mean = mean(imp_data), sd = sd(imp_data)), add=T, col = 4) print(density(imp_data)) plot(density(imp_data, bw=h), main = "Density Distribution of BsmtUnfSF") qqnorm(imp_data, col = 3, main = paste("QQnorm of BsmtUnfSF")) qqline(imp_data, col = 5, pch = 10) #' Possiamo dedurre che la distribuzione è positivamente asimetrica #' e che la maggior parte degli edifici ha un semiterrato di area compresa #' tra 810.5 e 1309.5 piedi^2, essendo rispettivamente il 1° e il 3° quantile #' e con maggior densita alineati alla retta qq della norma. #' Rimangono ancora degli outlier

readline("Press enter to continue: ")

#' Rimozione degli outlier attraverso intervallo di confidenza 95% per avvicinare alla norma mean_data = mean(data) sd_data = sd(data) confidence_level = 0.95 z_score = qnorm((1 + confidence_level) / 2)

lower_bound = mean_data - z_score * sd_data upper_bound = mean_data + z_score * sd_data

data_in_ci = data[data >= lower_bound & data <= upper_bound] print(list(summary = summary(data_in_ci), Variance = var(data_in_ci), Standard_deviation = sd(data_in_ci))) boxplot(data_in_ci, horizontal = T, main = paste("Boxplot of TotalBsmtSF")) h = 0.9(min(sd(data_in_ci), IQR(data_in_ci))/1.34)NROW(data_in_ci)^(-1/5) br = (max(data_in_ci)-min(data_in_ci))/h print(paste("br_CI = ",br)) hist(data_in_ci, probability = T, col = 2, main = paste("Hist of TotalBsmtSF"), breaks = br) curve(dnorm(x, mean = mean(data_in_ci), sd = sd(data_in_ci)), add=T, col = 4) print(density(data_in_ci)) plot(density(data_in_ci, bw=h), main = "Density Distribution of TotalBsmtSF") qqnorm(data_in_ci, col = 3, main = paste("QQnorm of TotalBsmtSF")) qqline(data_in_ci, col = 5, pch = 10) #' Possiamo notare che la maggior parte degli edifici ha un semiterrato di area compresa #' tra 804.0 e 1266.8 piedi^2, essendo rispettivamente il 1° e il 3° quantile #' e con maggior densita alineati alla retta qq della norma.

} visualization_TotalBsmtSF(na.omit(House_prices[,"TotalBsmtSF"]), sum_info_h$TotalBsmtSF)

#' ----------------------------- #' 10: Visualizazione X1stFlrSF #' -----------------------------

visualization_X1stFlrSF <- function(data, sum_info_h){ par(mfrow = c(2,2)) print(sum_info_h) boxplot(data, horizontal = TRUE, main = "Boxplot of X1stFlrSF") h = 0.9 * (min(sd(data), IQR(data)) / 1.34) * NROW(data)^(-1/5) br = (max(data) - min(data)) / h hist(data, probability = TRUE, col = 2, main = "Hist of X1stFlrSF", breaks = br) curve(dnorm(x, mean = mean(data), sd = sd(data)), add = TRUE, col = 4) print(density(data)) plot(density(data, bw = h), main = "Density Distribution of X1stFlrSF") qqnorm(data, col = 3, main = "QQnorm of X1stFlrSF") qqline(data, col = 5, pch = 10) #' Possiamo dedurre che la distribuzione è positivamente asimetrica #' e che la maggior parte degli edifici ha il primo piano di area compresa #' tra 882 e 1391 piedi^2, essendo rispettivamente il 1° e il 3° quantile #' e con maggior densita alineati alla retta qq della norma. #' Rimangono ancora degli outlier

readline("Press enter to continue: ")

#' Rimozione degli outlier attraverso intervallo di confidenza 95% per avvicinare alla norma mean_data = mean(data) sd_data = sd(data) confidence_level = 0.95 z_score = qnorm((1 + confidence_level) / 2)

lower_bound = mean_data - z_score * sd_data upper_bound = mean_data + z_score * sd_data

data_in_ci = data[data >= lower_bound & data <= upper_bound] print(list(summary = summary(data_in_ci), Variance = var(data_in_ci), Standard_deviation = sd(data_in_ci))) boxplot(data_in_ci, horizontal = T, main = paste("Boxplot of X1stFlrSF")) h = 0.9(min(sd(data_in_ci), IQR(data_in_ci))/1.34)NROW(data_in_ci)^(-1/5) br = (max(data_in_ci)-min(data_in_ci))/h print(paste("br_CI = ",br)) hist(data_in_ci, probability = T, col = 2, main = paste("Hist of X1stFlrSF"), breaks = br) curve(dnorm(x, mean = mean(data_in_ci), sd = sd(data_in_ci)), add=T, col = 4) print(density(data_in_ci)) plot(density(data_in_ci, bw=h), main = "Density Distribution of X1stFlrSF") qqnorm(data_in_ci, col = 3, main = paste("QQnorm of X1stFlrSF")) qqline(data_in_ci, col = 5, pch = 10) #' Possiamo notare che la maggior parte degli edifici ha il primo piano di area compresa #' tra 874 e 1344 piedi^2, essendo rispettivamente il 1° e il 3° quantile #' e con maggior densita alineati alla retta qq della norma. } visualization_X1stFlrSF(na.omit(House_prices[,"X1stFlrSF"]), sum_info_h$X1stFlrSF)

#' ----------------------------- #' 11: Visualizazione X2ndFlrSF #' -----------------------------

visualization_X2ndFlrSF <- function(data, sum_info_h){ par(mfrow = c(2,2)) print(sum_info_h) boxplot(data, horizontal = TRUE, main = "Boxplot of X2ndFlrSF") h = 0.9 * (min(sd(data), IQR(data)) / 1.34) * NROW(data)^(-1/5) br = (max(data) - min(data)) / h hist(data, probability = TRUE, col = 2, main = "Hist of X2ndFlrSF", breaks = br) curve(dnorm(x, mean = mean(data), sd = sd(data)), add = TRUE, col = 4) print(density(data)) plot(density(data, bw=h), main = "Density Distribution of X2ndFlrSF") qqnorm(data, col = 3, main = "QQnorm of X2ndFlrSF") qqline(data, col = 5, pch = 10) #' Possiamo dedurre che la distribuzione è positivamente asimetrica. #' Si puo osservare che l'area del secondo piano è prelevantemete 0 #' qundi non costruitto. #' Per fare un analisi sugli edifici che hanno il secondo piano, #' serve ignorare quelli senza

readline("Press enter to continue: ")

#' Rimozione dello 0 imp_data = data imp_data[imp_data == 0] = NA imp_data = na.omit(imp_data) print(list(summary = summary(imp_data), Variance = var(imp_data), Standard_deviation = sd(imp_data))) boxplot(imp_data, horizontal = T, main = paste("Boxplot of X2ndFlrSF")) h = 0.9(min(sd(imp_data), IQR(imp_data))/1.34)NROW(imp_data)^(-1/5) br = (max(imp_data)-min(imp_data))/h hist(imp_data, probability = T, col = 2, main = paste("Hist of X2ndFlrSF"), breaks = br) curve(dnorm(x, mean = mean(imp_data), sd = sd(imp_data)), add=T, col = 4) print(density(imp_data)) plot(density(imp_data, bw=h), main = "Density Distribution of X2ndFlrSF") qqnorm(imp_data, col = 3, main = paste("QQnorm of X2ndFlrSF")) qqline(imp_data, col = 5, pch = 10) #' Possiamo dedurre che la distribuzione è positivamente asimetrica #' e che la maggior parte degli edifici ha il secondo piano di area compresa #' tra 625.0 e 926.5 piedi^2, essendo rispettivamente il 1° e il 3° quantile #' e con maggior densita alineati alla retta qq della norma. #' Rimangono ancora degli outlier

readline("Press enter to continue: ")

#' Rimozione degli outlier attraverso intervallo di confidenza 90% per avvicinare alla norma mean_data = mean(imp_data) sd_data = sd(imp_data) confidence_level = 0.90 z_score = qnorm((1 + confidence_level) / 2)

lower_bound = mean_data - z_score * sd_data upper_bound = mean_data + z_score * sd_data

data_in_ci = imp_data[imp_data >= lower_bound & imp_data <= upper_bound] print(list(summary = summary(data_in_ci), Variance = var(data_in_ci), Standard_deviation = sd(data_in_ci))) boxplot(data_in_ci, horizontal = T, main = paste("Boxplot of X2ndFlrSF")) h = 0.9(min(sd(data_in_ci), IQR(data_in_ci))/1.34)NROW(data_in_ci)^(-1/5) br = (max(data_in_ci)-min(data_in_ci))/h print(paste("br_CI = ",br)) hist(data_in_ci, probability = T, col = 2, main = paste("Hist of X2ndFlrSF"), breaks = br) curve(dnorm(x, mean = mean(data_in_ci), sd = sd(data_in_ci)), add=T, col = 4) print(density(data_in_ci)) plot(density(data_in_ci, bw=h), main = "Density Distribution of X2ndFlrSF") qqnorm(data_in_ci, col = 3, main = paste("QQnorm of X2ndFlrSF")) qqline(data_in_ci, col = 5, pch = 10) #' Possiamo notare che la maggior parte degli edifici ha il secondo piano di area compresa #' tra 634 e 895 piedi^2, essendo rispettivamente il 1° e il 3° quantile #' e con maggior densita alineati alla retta qq della norma. } visualization_X2ndFlrSF(na.omit(House_prices[,"X2ndFlrSF"]), sum_info_h$X2ndFlrSF)

#' ----------------------------- #' 12: Visualizazione LowQualFinSF #' -----------------------------

visualization_LowQualFinSF <- function(data, sum_info_h){ par(mfrow = c(2,2)) print(sum_info_h) boxplot(data, horizontal = TRUE, main = "Boxplot of LowQualFinSF") h = 0.9 * (min(sd(data), IQR(data)) / 1.34) * NROW(data)^(-1/5) br = (max(data) - min(data)) / h hist(data, probability = TRUE, col = 2, main = "Hist of LowQualFinSF") curve(dnorm(x, mean = mean(data), sd = sd(data)), add = TRUE, col = 4) print(density(data)) plot(density(data), main = "Density Distribution of LowQualFinSF") qqnorm(data, col = 3, main = "QQnorm of LowQualFinSF") qqline(data, col = 5, pch = 10) #' Possiamo dedurre che la distribuzione è positivamente asimetrica. #' Si puo osservare che l'area dei piani fatti male è prelevantemete 0 #' Per fare un analisi sugli edifici che hanno i piani fatti male, #' serve ignorare quelli senza

readline("Press enter to continue: ")

#' Rimozione dello 0 imp_data = data imp_data[imp_data == 0] = NA imp_data = na.omit(imp_data) print(list(summary = summary(imp_data), Variance = var(imp_data), Standard_deviation = sd(imp_data))) boxplot(imp_data, horizontal = T, main = paste("Boxplot of LowQualFinSF")) h = 0.9(min(sd(imp_data), IQR(imp_data))/1.34)NROW(imp_data)^(-1/5) br = (max(imp_data)-min(imp_data))/h print(paste("br = ",br)) print(paste("len_data = ",length(data))) print(paste("len_data_wth_0 = ",length(imp_data))) print(paste("ratio_len_data_to_data = ", length(imp_data)/length(data))) hist(imp_data, probability = T, col = 2, main = paste("Hist of LowQualFinSF")) curve(dnorm(x, mean = mean(imp_data), sd = sd(imp_data)), add=T, col = 4) print(density(imp_data)) plot(density(imp_data), main = "Density Distribution of LowQualFinSF") qqnorm(imp_data, col = 3, main = paste("QQnorm of LowQualFinSF")) qqline(imp_data, col = 5, pch = 10) #' Possiamo dedurre che la distribuzione è positivamente asimetrica #' e che la maggior parte degli edifici ha i piani fatti male di area compresa #' tra 168.2 e 477.5 piedi^2, essendo rispettivamente il 1° e il 3° quantile #' e con maggior densita alineati alla retta qq della norma. #' Gli edifici che hanno i piani fatti male sono cosi piccoli (rispetto al totale ~=1,8%) #' che si puo trascurare } visualization_LowQualFinSF(na.omit(House_prices[,"LowQualFinSF"]), sum_info_h$LowQualFinSF)

#' ----------------------------- #' 13: Visualizazione GrLivArea #' -----------------------------

visualization_GrLivArea <- function(data, sum_info_h){ par(mfrow = c(2,2)) print(sum_info_h) boxplot(data, horizontal = TRUE, main = "Boxplot of GrLivArea") h = 0.9 * (min(sd(data), IQR(data)) / 1.34) * NROW(data)^(-1/5) br = (max(data) - min(data)) / h hist(data, probability = TRUE, col = 2, main = "Hist of GrLivArea", breaks = br) curve(dnorm(x, mean = mean(data), sd = sd(data)), add = TRUE, col = 4) print(density(data)) plot(density(data, bw=h), main = "Density Distribution of GrLivArea") qqnorm(data, col = 3, main = "QQnorm of GrLivArea") qqline(data, col = 5, pch = 10) #' Possiamo dedurre che la distribuzione è positivamente asimetrica #' e che la maggior parte degli edifici che sono sopra il seminterrato #' sono di area compresa tra 1130 e 1777 piedi^2, essendo rispettivamente #' il 1° e il 3° quantile e con maggior densita alineati alla retta qq della norma. #' Rimangono ancora degli outlier

readline("Press enter to continue: ")

#' Rimozione degli outlier attraverso intervallo di confidenza 95% per avvicinare alla norma mean_data = mean(data) sd_data = sd(data) confidence_level = 0.95 z_score = qnorm((1 + confidence_level) / 2)

lower_bound = mean_data - z_score * sd_data upper_bound = mean_data + z_score * sd_data

data_in_ci = data[data >= lower_bound & data <= upper_bound] print(list(summary = summary(data_in_ci), Variance = var(data_in_ci), Standard_deviation = sd(data_in_ci))) boxplot(data_in_ci, horizontal = T, main = paste("Boxplot of GrLivArea")) h = 0.9(min(sd(data_in_ci), IQR(data_in_ci))/1.34)NROW(data_in_ci)^(-1/5) br = (max(data_in_ci)-min(data_in_ci))/h print(paste("br_CI = ",br)) hist(data_in_ci, probability = T, col = 2, main = paste("Hist of GrLivArea"), breaks = br) curve(dnorm(x, mean = mean(data_in_ci), sd = sd(data_in_ci)), add=T, col = 4) print(density(data_in_ci)) plot(density(data_in_ci, bw=h), main = "Density Distribution of GrLivArea") qqnorm(data_in_ci, col = 3, main = paste("QQnorm of GrLivArea")) qqline(data_in_ci, col = 5, pch = 10) #' Possiamo notare che la maggior parte degli edifici che sono sopra il seminterrato #' sono compresi tra 1121 e 1728 piedi^2, essendo rispettivamente #' il 1° e il 3° quantile e con maggior densita alineati alla retta qq della norma. } visualization_GrLivArea(na.omit(House_prices[,"GrLivArea"]), sum_info_h$GrLivArea)

#' ----------------------------- #' 14: Visualizazione BsmtFullBath #' -----------------------------

visualization_BsmtFullBath <- function(data, sum_info_h){ par(mfrow = c(1,1)) print("freq: ") cat("\n") print(table(data)) cat("\n") print("percintile: ") cat("\n") print(prop.table(table(data))*100) barplot(table(data), col = 2, main = "Barplot of BsmtFullBath") #' Si osserva che la maggior parte dei semiterrati non hanno un bagno completo essendo #' il ~=58.6% con frequenza di 856 #' Si osserva che la minima freq dei numeri di bagno completi e 1 per 3 bangi #' e quindi si puo trascurare essendo <1% #' Si osserva che la freq dei numeri di bango completi per 2 bangi e 15 che è ~= 1.03% e #' quindi anch'esso trascurabile #' Infine si osserva che la maggior parte dei semiterrati che hanno 1 bagno completo #' è ~=40.3% con frequenza di 588 } visualization_BsmtFullBath(na.omit(House_prices[,"BsmtFullBath"]), sum_info_h$BsmtFullBath)

#' ----------------------------- #' 15: Visualizazione BsmtHalfBath #' -----------------------------

visualization_BsmtHalfBath <- function(data, sum_info_h){ par(mfrow = c(1,1)) print("freq: ") cat("\n") print(table(data)) cat("\n") print("percintile: ") cat("\n") print(prop.table(table(data))*100) barplot(table(data), col = 2, main = "Barplot of BsmtHalfBath") #' Si osserva che la maggior parte dei semiterrati non hanno un bagno mezzo completo essendo #' il ~94% con frequenza di 1378 #' Si osserva che la freq dei numeri di bangi mezzi completi per 2 bangi e 2 che #' è ~= 1% e quindi anch'esso trascurabile #' Infine si osserva che la maggior parte dei semiterrati che hanno 1 bagno mezzo completo #' è ~5% con frequenza di 80 } visualization_BsmtHalfBath(na.omit(House_prices[,"BsmtHalfBath"]), sum_info_h$BsmtHalfBath)

#' ----------------------------- #' 16: Visualizazione FullBath #' -----------------------------

visualization_FullBath <- function(data, sum_info_h){ par(mfrow = c(1,1)) print("freq: ") cat("\n") print(table(data)) cat("\n") print("percintile: ") cat("\n") print(prop.table(table(data))*100) barplot(table(data), col = 2, main = "Barplot of FullBath") #' Si osserva che la maggior parte degli edifici che sono sopra il seminterrato #' hanno 2 bagni completi essendo il ~=52.6% con frequenza di 768 #' Si osserva che la minima freq dei numeri di bagno completi e 9 per nessun bango #' e quindi si puo trascurare essendo <1% #' Si osserva che la freq dei numeri di bango completi per 3 bangi e 33 che è ~= 2.2% e #' quindi anch'esso trascurabile #' Infine si osserva che gli edifici che sono sopra il seminterrato #' che hanno 1 bagno completo sono il ~=44.5% con frequenza di 650 } visualization_FullBath(na.omit(House_prices[,"FullBath"]), sum_info_h$FullBath)

#' ----------------------------- #' 17: Visualizazione HalfBath #' -----------------------------

visualization_HalfBath <- function(data, sum_info_h){ par(mfrow = c(1,1)) print("freq: ") cat("\n") print(table(data)) cat("\n") print("percintile: ") cat("\n") print(prop.table(table(data))*100) barplot(table(data), col = 2, main = "Barplot of HalfBath") #' Si osserva che la maggior parte degli edifici che sono sopra il seminterrato #' hanno 0 bagni mezzo completi essendo il ~=62.5% con frequenza di 913 #' Si osserva che la minima freq dei numeri di bagno completi e 12 per 2 bango #' e quindi si puo trascurare essendo <1% #' Infine si osserva che gli edifici che sono sopra il seminterrato #' che hanno 1 bagno completo sono il ~=36.6% con frequenza di 535 } visualization_HalfBath(na.omit(House_prices[,"HalfBath"]), sum_info_h$HalfBath)

#' ----------------------------- #' 18: Visualizazione BedroomAbvGr #' -----------------------------

visualization_BedroomAbvGr <- function(data, sum_info_h){ par(mfrow = c(1,1)) print("freq: ") cat("\n") print(table(data)) cat("\n") print("percintile: ") cat("\n") print(prop.table(table(data))*100) barplot(table(data), col = 2, main = "Barplot of BedroomAbvGr") #' Si osserva che la maggior parte degli edifici che sono sopra il seminterrato #' hanno 3 camere essendo il ~=55.07% con frequenza di 804 seguito da 2, 4 e 1 camere #' essendo ~=24.5%, ~=14.6% e ~=3.4% con frequenze 358, 213 e 50 rispettivamente. #' Si osserva che la minima freq dei numeri di camere e 1 per 8 camere seguito da #' 5, 6 e 0 camere con frequenze 21, 7 e 6 rispettivamente #' e quindi si puo trascurare essendo tutti <1.5% } visualization_BedroomAbvGr(na.omit(House_prices[,"BedroomAbvGr"]), sum_info_h$BedroomAbvGr)

#' ----------------------------- #' 19: Visualizazione KitchenAbvGr #' -----------------------------

visualization_KitchenAbvGr <- function(data, sum_info_h){ par(mfrow = c(1,1)) print("freq: ") cat("\n") print(table(data)) cat("\n") print("percintile: ") cat("\n") print(prop.table(table(data))*100) barplot(table(data), col = 2, main = "Barplot of KitchenAbvGr") #' Si osserva che la maggior parte degli edifici che sono sopra il seminterrato #' hanno 1 cucina essendo il ~=95.34% con frequenza di 1392 seguito da 2 cucine #' essendo ~=4.45% con frequenze 65. #' Si osserva che la minima freq dei numeri di camere e 1 per 0 cucine seguito da #' 3 cucine con frequenze 2 e quindi si puo trascurare essendo tutti <1% } visualization_KitchenAbvGr(na.omit(House_prices[,"KitchenAbvGr"]), sum_info_h$KitchenAbvGr)

#' ----------------------------- #' 20: Visualizazione TotRmsAbvGrd #' -----------------------------

visualization_TotRmsAbvGrd <- function(data, sum_info_h){ par(mfrow = c(2,2)) boxplot(data, horizontal = TRUE, main = "Boxplot of TotRmsAbvGrd") print("freq: ") cat("\n") print(table(data)) cat("\n") print("percintile: ") cat("\n") print(prop.table(table(data))*100) hist(data, probability = T, col = 2, main = "Hist of TotRmsAbvGrd") barplot(table(data), xlim=c(0,14), main = "Barplot of TotRmsAbvGrd") #' Si osserva che la maggior parte degli edifici che sono sopra il seminterrato #' sono nel intervallo fra 5 e 7 stanze (non considerando i bagni) essendo il ~=68.8% #' con il piu grande di occorenze 6 stanze (~=27.5%) con frequenza di 402. #' Si osserva che la minima freq dei numeri di stanze e 1 per 2 e 14 stanze seguito da #' 12, 3 e 11 stanze con frequenze 11, 17 e 18 rispettivamente e quindi si puo #' trascurare essendo tutti <1.5% } visualization_TotRmsAbvGrd(na.omit(House_prices[,"TotRmsAbvGrd"]), sum_info_h$TotRmsAbvGrd)

#' ----------------------------- #' 21: Visualizazione Fireplaces #' -----------------------------

visualization_Fireplaces <- function(data, sum_info_h){ par(mfrow = c(1,1)) print("freq: ") cat("\n") print(table(data)) cat("\n") print("percintile: ") cat("\n") print(prop.table(table(data))*100) barplot(table(data), col = 2, main = "Barplot of Fireplaces") #' Si osserva che la maggior parte degli edifici hanno 0 camini #' essendo il ~=47.26% con frequenza di 690 seguito da 1 e 2 camini #' essendo ~=44.52% con frequenze 650 e ~=7.88% con frequenza 115 rispettivamente. #' Si osserva che la minima freq dei numeri di camini e 5 per 3 camini #' e quindi si puo trascurare essendo <1% } visualization_Fireplaces(na.omit(House_prices[,"Fireplaces"]), sum_info_h$Fireplaces)

#' ----------------------------- #' 22: Visualizazione GarageYrBlt #' -----------------------------

visualization_GarageYrBlt <- function(data, sum_info_h){ par(mfrow = c(2,2)) print(sum_info_h) boxplot(data, horizontal = TRUE, main = "Boxplot of GarageYrBlt") h = 0.9 * (min(sd(data), IQR(data)) / 1.34) * NROW(data)^(-1/5) br = (max(data) - min(data)) / h print(paste("br = ",br)) hist(data, probability = F, col = 2, main = "Hist of GarageYrBlt", breaks = br) print(density(data)) plot(density(data), main = "Density of GarageYrBlt") #' Possiamo dedurre che la distribuzione è positivamente asimetrica #' e quindi il numero di garage costruiti crescono ogni anno #' La maggior parte degli garage erano costruiti fra 1961 e 2002, #' essendo rispettivamente il 1° e il 3° quantile e l'anno con #' è il massimo costruitto è nel 2005 con 65 garage } visualization_GarageYrBlt(na.omit(House_prices[,"GarageYrBlt"]), sum_info_h$GarageYrBlt)

#' ----------------------------- #' 23: Visualizazione GarageCars #' -----------------------------

visualization_GarageCars <- function(data, sum_info_h){ par(mfrow = c(1,1)) print("freq: ") cat("\n") print(table(data)) cat("\n") print("percintile: ") cat("\n") print(prop.table(table(data))*100) barplot(table(data), col = 2, main = "Barplot of GarageCars") #' Si osserva che la maggior parte degli Garage possono avere 2 auto #' essendo il ~=56.44% con frequenza di 824 seguito da 1, 3 e 0 auto #' essendo ~=25.27% con frequenze 369, ~=12.4% con frequenze 181 #' e ~=5.55% con frequenze 81 rispettivamente. #' Si osserva che la minima freq dei numeri di auto e 5 per 4 auto #' e quindi si puo trascurare essendo <1% } visualization_GarageCars(na.omit(House_prices[,"GarageCars"]), sum_info_h$GarageCars)

#' ----------------------------- #' 23: Visualizazione GarageArea #' -----------------------------

visualization_GarageArea <- function(data, sum_info_h){ par(mfrow = c(2,2)) print(sum_info_h) boxplot(data, horizontal = TRUE, main = "Boxplot of GarageArea") h = 0.9 * (min(sd(data), IQR(data)) / 1.34) * NROW(data)^(-1/5) br = (max(data) - min(data)) / h hist(data, probability = TRUE, col = 2, main = "Hist of GarageArea") curve(dnorm(x, mean = mean(data), sd = sd(data)), add = TRUE, col = 4) print(density(data)) plot(density(data), main = "Distribution of GarageArea with smoothing parameter bw") qqnorm(data, col = 3, main = "QQnorm of GarageArea") qqline(data, col = 5, pch = 10) #' Possiamo dedurre che la distribuzione è positivamente asimetrica #' e che la maggior parte degli garage sono di area compresa tra 334.5 e 576.0 piedi^2, #' essendo rispettivamente il 1° e il 3° quantile e #' con maggior densita alineati alla retta qq della norma. #' Rimangono ancora degli outlier

readline("Press enter to continue: ")

#' Rimozione degli outlier attraverso intervallo di confidenza 90% per avvicinare alla norma mean_data = mean(data) sd_data = sd(data) confidence_level = 0.90 z_score = qnorm((1 + confidence_level) / 2)

lower_bound = mean_data - z_score * sd_data upper_bound = mean_data + z_score * sd_data

data_in_ci = data[data >= lower_bound & data <= upper_bound] print(list(summary = summary(data_in_ci), Variance = var(data_in_ci), Standard_deviation = sd(data_in_ci))) boxplot(data_in_ci, horizontal = T, main = paste("Boxplot of GrLivArea")) h = 0.9(min(sd(data_in_ci), IQR(data_in_ci))/1.34)NROW(data_in_ci)^(-1/5) br = (max(data_in_ci)-min(data_in_ci))/h print(paste("br_CI = ",br)) hist(data_in_ci, probability = T, col = 2, main = paste("Hist of GrLivArea")) curve(dnorm(x, mean = mean(data_in_ci), sd = sd(data_in_ci)), add=T, col = 4) print(density(data_in_ci)) plot(density(data_in_ci), main = "Density Distribution of GrLivArea") qqnorm(data_in_ci, col = 3, main = paste("QQnorm of GrLivArea")) qqline(data_in_ci, col = 5, pch = 10) #' Possiamo notare che la maggior parte degli garage #' sono compresi tra 358.0 e 566.0 piedi^2, essendo rispettivamente #' il 1° e il 3° quantile e con maggior densita alineati alla retta qq della norma. } visualization_GarageArea(na.omit(House_prices[,"GarageArea"]), sum_info_h$GarageArea)

#' ----------------------------- #' 24: Visualizazione WoodDeckSF #' -----------------------------

visualization_WoodDeckSF <- function(data, sum_info_h){ par(mfrow = c(2,2)) print(sum_info_h) boxplot(data, horizontal = TRUE, main = "Boxplot of WoodDeckSF") h = 0.9 * (min(sd(data), IQR(data)) / 1.34) * NROW(data)^(-1/5) br = (max(data) - min(data)) / h hist(data, probability = TRUE, col = 2, main = "Hist of WoodDeckSF") curve(dnorm(x, mean = mean(data), sd = sd(data)), add = TRUE, col = 4) print(density(data)) plot(density(data), main = "Density Distribution of WoodDeckSF") qqnorm(data, col = 3, main = "QQnorm of WoodDeckSF") qqline(data, col = 5, pch = 10) #' Possiamo dedurre che la distribuzione è positivamente asimetrica. #' Si puo osservare che l'area del ponte di legno è prelevantemete 0 con 761 occorrenze #' qundi non costruitto. #' Per fare un analisi sui ponti di legno costruiti, serve ignorare quelli non

readline("Press enter to continue: ")

#' Rimozione dello 0 imp_data = data imp_data[imp_data == 0] = NA imp_data = na.omit(imp_data) print(list(summary = summary(imp_data), Variance = var(imp_data), Standard_deviation = sd(imp_data))) boxplot(imp_data, horizontal = T, main = paste("Boxplot of WoodDeckSF")) h = 0.9(min(sd(imp_data), IQR(imp_data))/1.34)NROW(imp_data)^(-1/5) br = (max(imp_data)-min(imp_data))/h hist(imp_data, probability = T, col = 2, main = paste("Hist of WoodDeckSF"), breaks = br) curve(dnorm(x, mean = mean(imp_data), sd = sd(imp_data)), add=T, col = 4) print(density(imp_data)) plot(density(imp_data, bw=h), main = "Density Distribution of WoodDeckSF") qqnorm(imp_data, col = 3, main = paste("QQnorm of WoodDeckSF")) qqline(imp_data, col = 5, pch = 10) #' Possiamo dedurre che la distribuzione è positivamente asimetrica #' e che la maggior parte dei ponte di legno sono di area compresa #' tra 120.0 e 240.0 piedi^2, essendo rispettivamente il 1° e il 3° quantile #' e con maggior densita alineati alla retta qq della norma. #' Rimangono ancora degli outlier

readline("Press enter to continue: ")

#' Rimozione degli outlier attraverso intervallo di confidenza 88% per avvicinare alla norma mean_data = mean(imp_data) sd_data = sd(imp_data) confidence_level = 0.88 z_score = qnorm((1 + confidence_level) / 2)

lower_bound = mean_data - z_score * sd_data upper_bound = mean_data + z_score * sd_data

data_in_ci = imp_data[imp_data >= lower_bound & imp_data <= upper_bound] print(list(summary = summary(data_in_ci), Variance = var(data_in_ci), Standard_deviation = sd(data_in_ci))) boxplot(data_in_ci, horizontal = T, main = paste("Boxplot of WoodDeckSF")) h = 0.9(min(sd(data_in_ci), IQR(data_in_ci))/1.34)NROW(data_in_ci)^(-1/5) br = (max(data_in_ci)-min(data_in_ci))/h print(paste("br_CI = ",br)) hist(data_in_ci, probability = T, col = 2, main = paste("Hist of WoodDeckSF"), breaks = br) curve(dnorm(x, mean = mean(data_in_ci), sd = sd(data_in_ci)), add=T, col = 4) print(density(data_in_ci)) plot(density(data_in_ci, bw=h), main = "Density Distribution of WoodDeckSF") qqnorm(data_in_ci, col = 3, main = paste("QQnorm of WoodDeckSF")) qqline(data_in_ci, col = 5, pch = 10) #' Possiamo notare che la maggior parte dei ponte di legno sono di area compresa #' tra 120.0 e 220.5 piedi^2, essendo rispettivamente il 1° e il 3° quantile #' e con maggior densita alineati alla retta qq della norma. } visualization_WoodDeckSF(na.omit(House_prices[,"WoodDeckSF"]), sum_info_h$WoodDeckSF)

#' ----------------------------- #' 25: Visualizazione OpenPorchSF #' -----------------------------

visualization_OpenPorchSF <- function(data, sum_info_h){ par(mfrow = c(2,2)) print(sum_info_h) boxplot(data, horizontal = TRUE, main = "Boxplot of OpenPorchSF") h = 0.9 * (min(sd(data), IQR(data)) / 1.34) * NROW(data)^(-1/5) br = (max(data) - min(data)) / h hist(data, probability = TRUE, col = 2, main = "Hist of OpenPorchSF") curve(dnorm(x, mean = mean(data), sd = sd(data)), add = TRUE, col = 4) print(density(data)) plot(density(data), main = "Distribution of OpenPorchSF with smoothing parameter bw") qqnorm(data, col = 3, main = "QQnorm of OpenPorchSF") qqline(data, col = 5, pch = 10) #' Possiamo dedurre che la distribuzione è positivamente asimetrica. #' Si puo osservare che l'area del portico aperto è prelevantemete 0 con 656 occorrenze #' qundi non costruitto. #' Per fare un analisi sui portico aperti, serve ignorare quelli non

readline("Press enter to continue: ")

#' Rimozione dello 0 imp_data = data imp_data[imp_data == 0] = NA imp_data = na.omit(imp_data) print(list(summary = summary(imp_data), Variance = var(imp_data), Standard_deviation = sd(imp_data))) boxplot(imp_data, horizontal = T, main = paste("Boxplot of WoodDeckSF")) h = 0.9(min(sd(imp_data), IQR(imp_data))/1.34)NROW(imp_data)^(-1/5) br = (max(imp_data)-min(imp_data))/h hist(imp_data, probability = T, col = 2, main = paste("Hist of WoodDeckSF"), breaks = br) curve(dnorm(x, mean = mean(imp_data), sd = sd(imp_data)), add=T, col = 4) print(density(imp_data)) plot(density(imp_data, bw=h), main = "Density Distribution of WoodDeckSF") qqnorm(imp_data, col = 3, main = paste("QQnorm of WoodDeckSF")) qqline(imp_data, col = 5, pch = 10) #' Possiamo dedurre che la distribuzione è positivamente asimetrica #' e che la maggior parte dei portico aperti sono di area compresa #' tra 39.00 e 112.00 piedi^2, essendo rispettivamente il 1° e il 3° quantile #' e con maggior densita alineati alla retta qq della norma. #' Rimangono ancora degli outlier

readline("Press enter to continue: ")

#' Rimozione degli outlier attraverso intervallo di confidenza 88% per avvicinare alla norma mean_data = mean(imp_data) sd_data = sd(imp_data) confidence_level = 0.88 z_score = qnorm((1 + confidence_level) / 2)

lower_bound = mean_data - z_score * sd_data upper_bound = mean_data + z_score * sd_data

data_in_ci = imp_data[imp_data >= lower_bound & imp_data <= upper_bound] print(list(summary = summary(data_in_ci), Variance = var(data_in_ci), Standard_deviation = sd(data_in_ci))) boxplot(data_in_ci, horizontal = T, main = paste("Boxplot of WoodDeckSF")) h = 0.9(min(sd(data_in_ci), IQR(data_in_ci))/1.34)NROW(data_in_ci)^(-1/5) br = (max(data_in_ci)-min(data_in_ci))/h print(paste("br_CI = ",br)) hist(data_in_ci, probability = T, col = 2, main = paste("Hist of WoodDeckSF"), breaks = br) curve(dnorm(x, mean = mean(data_in_ci), sd = sd(data_in_ci)), add=T, col = 4) print(density(data_in_ci)) plot(density(data_in_ci, bw=h), main = "Density Distribution of WoodDeckSF") qqnorm(data_in_ci, col = 3, main = paste("QQnorm of WoodDeckSF")) qqline(data_in_ci, col = 5, pch = 10) #' Possiamo notare che la maggior parte dei portico aperti sono di area compresa #' tra 120.0 e 220.5 piedi^2, essendo rispettivamente il 1° e il 3° quantile #' e con maggior densita alineati alla retta qq della norma. } visualization_OpenPorchSF(na.omit(House_prices[,"OpenPorchSF"]), sum_info_h$OpenPorchSF)

#' ----------------------------- #' 26: Visualizazione EnclosedPorch #' -----------------------------

visualization_EnclosedPorch <- function(data, sum_info_h){ par(mfrow = c(2,2)) print(sum_info_h) boxplot(data, horizontal = TRUE, main = "Boxplot of EnclosedPorch") h = 0.9 * (min(sd(data), IQR(data)) / 1.34) * NROW(data)^(-1/5) br = (max(data) - min(data)) / h hist(data, probability = TRUE, col = 2, main = "Hist of EnclosedPorch") curve(dnorm(x, mean = mean(data), sd = sd(data)), add = TRUE, col = 4) print(density(data)) plot(density(data), main = "Density Distribution of EnclosedPorch") qqnorm(data, col = 3, main = "QQnorm of EnclosedPorch") qqline(data, col = 5, pch = 10) #' Possiamo dedurre che la distribuzione è positivamente asimetrica. #' Si puo osservare che l'area del portico chiuso è prelevantemete 0 con 1252 occorrenze #' qundi non costruitto. #' Per fare un analisi sui portico chiusi, serve ignorare quelli non

readline("Press enter to continue: ")

#' Rimozione dello 0 imp_data = data imp_data[imp_data == 0] = NA imp_data = na.omit(imp_data) print(list(summary = summary(imp_data), Variance = var(imp_data), Standard_deviation = sd(imp_data))) boxplot(imp_data, horizontal = T, main = paste("Boxplot of EnclosedPorch")) h = 0.9(min(sd(imp_data), IQR(imp_data))/1.34)NROW(imp_data)^(-1/5) br = (max(imp_data)-min(imp_data))/h hist(imp_data, probability = T, col = 2, main = paste("Hist of EnclosedPorch"), breaks = br) curve(dnorm(x, mean = mean(imp_data), sd = sd(imp_data)), add=T, col = 4) print(density(imp_data)) plot(density(imp_data, bw=h), main = "Density Distribution of EnclosedPorch") qqnorm(imp_data, col = 3, main = paste("QQnorm of EnclosedPorch")) qqline(imp_data, col = 5, pch = 10) #' Possiamo dedurre che la distribuzione è positivamente asimetrica #' e che la maggior parte dei portico chiusi sono di area compresa #' tra 104.2 e 205.0 piedi^2, essendo rispettivamente il 1° e il 3° quantile #' e con maggior densita alineati alla retta qq della norma. #' Rimangono ancora degli outlier

readline("Press enter to continue: ")

#' Rimozione degli outlier attraverso intervallo di confidenza 95% per avvicinare alla norma mean_data = mean(imp_data) sd_data = sd(imp_data) confidence_level = 0.95 z_score = qnorm((1 + confidence_level) / 2)

lower_bound = mean_data - z_score * sd_data upper_bound = mean_data + z_score * sd_data

data_in_ci = imp_data[imp_data >= lower_bound & imp_data <= upper_bound] print(list(summary = summary(data_in_ci), Variance = var(data_in_ci), Standard_deviation = sd(data_in_ci))) boxplot(data_in_ci, horizontal = T, main = paste("Boxplot of EnclosedPorch")) h = 0.9(min(sd(data_in_ci), IQR(data_in_ci))/1.34)NROW(data_in_ci)^(-1/5) br = (max(data_in_ci)-min(data_in_ci))/h print(paste("br_CI = ",br)) hist(data_in_ci, probability = T, col = 2, main = paste("Hist of EnclosedPorch"), breaks = br) curve(dnorm(x, mean = mean(data_in_ci), sd = sd(data_in_ci)), add=T, col = 4) print(density(data_in_ci)) plot(density(data_in_ci, bw=h), main = "Density Distribution of EnclosedPorch") qqnorm(data_in_ci, col = 3, main = paste("QQnorm of EnclosedPorch")) qqline(data_in_ci, col = 5, pch = 10) #' Possiamo notare che la maggior parte dei portico chiusi sono di area compresa #' tra 102.0 e 200.5 piedi^2, essendo rispettivamente il 1° e il 3° quantile #' e con maggior densita alineati alla retta qq della norma. } visualization_EnclosedPorch(na.omit(House_prices[,"EnclosedPorch"]), sum_info_h$EnclosedPorch)

#' ----------------------------- #' 27: Visualizazione X3SsnPorch #' -----------------------------

visualization_X3SsnPorch <- function(data, sum_info_h){ par(mfrow = c(2,2)) print(sum_info_h) boxplot(data, horizontal = TRUE, main = "Boxplot of X3SsnPorch") h = 0.9 * (min(sd(data), IQR(data)) / 1.34) * NROW(data)^(-1/5) br = (max(data) - min(data)) / h hist(data, probability = TRUE, col = 2, main = "Hist of X3SsnPorch") curve(dnorm(x, mean = mean(data), sd = sd(data)), add = TRUE, col = 4) print(density(data)) plot(density(data), main = "Density Distribution of X3SsnPorch") qqnorm(data, col = 3, main = "QQnorm of X3SsnPorch") qqline(data, col = 5, pch = 10) #' Possiamo dedurre che la distribuzione è positivamente asimetrica. #' Si puo osservare che l'area del Portico per tre stagioni è prelevantemete #' 0 con 1252 occorrenze qundi non costruitto. #' Per fare un analisi sui Portico per tre stagioni, serve ignorare quelli non

readline("Press enter to continue: ")

#' Rimozione dello 0 imp_data = data imp_data[imp_data == 0] = NA imp_data = na.omit(imp_data) print(list(summary = summary(imp_data), Variance = var(imp_data), Standard_deviation = sd(imp_data))) boxplot(imp_data, horizontal = T, main = paste("Boxplot of X3SsnPorch")) h = 0.9(min(sd(imp_data), IQR(imp_data))/1.34)NROW(imp_data)^(-1/5) br = (max(imp_data)-min(imp_data))/h hist(imp_data, probability = T, col = 2, main = paste("Hist of X3SsnPorch"), breaks = br) curve(dnorm(x, mean = mean(imp_data), sd = sd(imp_data)), add=T, col = 4) print(density(imp_data)) plot(density(imp_data, bw=h), main = "Density Distribution of X3SsnPorch") qqnorm(imp_data, col = 3, main = paste("QQnorm of X3SsnPorch")) qqline(imp_data, col = 5, pch = 10) #' Possiamo dedurre che la distribuzione è positivamente asimetrica #' e che la maggior parte dei Portico per tre stagioni sono di area compresa #' tra 150.8 e 239.8 piedi^2, essendo rispettivamente il 1° e il 3° quantile #' e con maggior densita alineati alla retta qq della norma. #' Rimangono ancora degli outlier

readline("Press enter to continue: ")

#' Rimozione degli outlier attraverso intervallo di confidenza 95% per avvicinare alla norma mean_data = mean(imp_data) sd_data = sd(imp_data) confidence_level = 0.95 z_score = qnorm((1 + confidence_level) / 2)

lower_bound = mean_data - z_score * sd_data upper_bound = mean_data + z_score * sd_data

data_in_ci = imp_data[imp_data >= lower_bound & imp_data <= upper_bound] print(list(summary = summary(data_in_ci), Variance = var(data_in_ci), Standard_deviation = sd(data_in_ci))) boxplot(data_in_ci, horizontal = T, main = paste("Boxplot of X3SsnPorch")) h = 0.9(min(sd(data_in_ci), IQR(data_in_ci))/1.34)NROW(data_in_ci)^(-1/5) br = (max(data_in_ci)-min(data_in_ci))/h print(paste("br_CI = ",br)) hist(data_in_ci, probability = T, col = 2, main = paste("Hist of X3SsnPorch"), breaks = br) curve(dnorm(x, mean = mean(data_in_ci), sd = sd(data_in_ci)), add=T, col = 4) print(density(data_in_ci)) plot(density(data_in_ci, bw=h), main = "Density Distribution of X3SsnPorch") qqnorm(data_in_ci, col = 3, main = paste("QQnorm of X3SsnPorch")) qqline(data_in_ci, col = 5, pch = 10) #' Possiamo notare che la maggior parte dei Portico per tre stagioni sono di area compresa #' tra 146.2 e 216.0 piedi^2, essendo rispettivamente il 1° e il 3° quantile #' e con maggior densita alineati alla retta qq della norma. } visualization_X3SsnPorch(na.omit(House_prices[,"X3SsnPorch"]), sum_info_h$X3SsnPorch)

#' ----------------------------- #' 28: Visualizazione ScreenPorch #' -----------------------------

visualization_ScreenPorch <- function(data, sum_info_h){ par(mfrow = c(2,2)) print(sum_info_h) boxplot(data, horizontal = TRUE, main = "Boxplot of ScreenPorch") h = 0.9 * (min(sd(data), IQR(data)) / 1.34) * NROW(data)^(-1/5) br = (max(data) - min(data)) / h hist(data, probability = TRUE, col = 2, main = "Hist of ScreenPorch") curve(dnorm(x, mean = mean(data), sd = sd(data)), add = TRUE, col = 4) print(density(data)) plot(density(data), main = "Density Distribution of ScreenPorch") qqnorm(data, col = 3, main = "QQnorm of ScreenPorch") qqline(data, col = 5, pch = 10) #' Possiamo dedurre che la distribuzione è positivamente asimetrica. #' Si puo osservare che l'area del Portico dello schermo è prelevantemete #' 0 con 1344 occorrenze qundi non costruitto. #' Per fare un analisi sui Portico dello schermo, serve ignorare quelli non

readline("Press enter to continue: ")

#' Rimozione dello 0 imp_data = data imp_data[imp_data == 0] = NA imp_data = na.omit(imp_data) print(list(summary = summary(imp_data), Variance = var(imp_data), Standard_deviation = sd(imp_data))) boxplot(imp_data, horizontal = T, main = paste("Boxplot of ScreenPorch")) h = 0.9(min(sd(imp_data), IQR(imp_data))/1.34)NROW(imp_data)^(-1/5) br = (max(imp_data)-min(imp_data))/h hist(imp_data, probability = T, col = 2, main = paste("Hist of ScreenPorch"), breaks = br) curve(dnorm(x, mean = mean(imp_data), sd = sd(imp_data)), add=T, col = 4) print(density(imp_data)) plot(density(imp_data, bw=h), main = "Density Distribution of ScreenPorch") qqnorm(imp_data, col = 3, main = paste("QQnorm of ScreenPorch")) qqline(imp_data, col = 5, pch = 10) #' Possiamo dedurre che la distribuzione è positivamente asimetrica #' e che la maggior parte dei Portico dello schermo stagioni sono di area compresa #' tra 143.8 e 224.0 piedi^2, essendo rispettivamente il 1° e il 3° quantile #' e con maggior densita alineati alla retta qq della norma. #' Rimangono ancora degli outlier

readline("Press enter to continue: ")

#' Rimozione degli outlier attraverso intervallo di confidenza 95% per avvicinare alla norma mean_data = mean(imp_data) sd_data = sd(imp_data) confidence_level = 0.95 z_score = qnorm((1 + confidence_level) / 2)

lower_bound = mean_data - z_score * sd_data upper_bound = mean_data + z_score * sd_data

data_in_ci = imp_data[imp_data >= lower_bound & imp_data <= upper_bound] print(list(summary = summary(data_in_ci), Variance = var(data_in_ci), Standard_deviation = sd(data_in_ci))) boxplot(data_in_ci, horizontal = T, main = paste("Boxplot of ScreenPorch")) h = 0.9(min(sd(data_in_ci), IQR(data_in_ci))/1.34)NROW(data_in_ci)^(-1/5) br = (max(data_in_ci)-min(data_in_ci))/h print(paste("br_CI = ",br)) hist(data_in_ci, probability = T, col = 2, main = paste("Hist of ScreenPorch"), breaks = br) curve(dnorm(x, mean = mean(data_in_ci), sd = sd(data_in_ci)), add=T, col = 4) print(density(data_in_ci)) plot(density(data_in_ci, bw=h), main = "Density Distribution of ScreenPorch") qqnorm(data_in_ci, col = 3, main = paste("QQnorm of ScreenPorch")) qqline(data_in_ci, col = 5, pch = 10) #' Possiamo notare che la maggior parte dei Portico dello schermo sono di area compresa #' tra 142.2 e 214.5 piedi^2, essendo rispettivamente il 1° e il 3° quantile #' e con maggior densita alineati alla retta qq della norma. } visualization_ScreenPorch(na.omit(House_prices[,"ScreenPorch"]), sum_info_h$ScreenPorch)

#' ----------------------------- #' 29: Visualizazione PoolArea #' -----------------------------

visualization_PoolArea <- function(data, sum_info_h){ par(mfrow = c(2,2)) print(sum_info_h) boxplot(data, horizontal = TRUE, main = "Boxplot of PoolArea") h = 0.9 * (min(sd(data), IQR(data)) / 1.34) * NROW(data)^(-1/5) br = (max(data) - min(data)) / h hist(data, probability = TRUE, col = 2, main = "Hist of PoolArea") curve(dnorm(x, mean = mean(data), sd = sd(data)), add = TRUE, col = 4) print(density(data)) plot(density(data), main = "Density Distribution of PoolArea") qqnorm(data, col = 3, main = "QQnorm of PoolArea") qqline(data, col = 5, pch = 10) #' Possiamo dedurre che la distribuzione è positivamente asimetrica. #' Si puo osservare che l'area della piscina è prelevantemete #' 0 con 1453 occorrenze qundi non costruitto. #' Per fare un analisi sulle piscine, serve ignorare quelli non

readline("Press enter to continue: ")

#' Rimozione dello 0 imp_data = data imp_data[imp_data == 0] = NA imp_data = na.omit(imp_data) print(list(summary = summary(imp_data), Variance = var(imp_data), Standard_deviation = sd(imp_data))) boxplot(imp_data, horizontal = T, main = paste("Boxplot of PoolArea")) h = 0.9(min(sd(imp_data), IQR(imp_data))/1.34)NROW(imp_data)^(-1/5) br = (max(imp_data)-min(imp_data))/h hist(imp_data, probability = T, col = 2, main = paste("Hist of PoolArea"), breaks = br) curve(dnorm(x, mean = mean(imp_data), sd = sd(imp_data)), add=T, col = 4) print(density(imp_data)) plot(density(imp_data, bw=h), main = "Density Distribution of PoolArea") qqnorm(imp_data, col = 3, main = paste("QQnorm of PoolArea")) qqline(imp_data, col = 5, pch = 10) #' Possiamo dedurre che la distribuzione è positivamente asimetrica #' e che la maggior parte delle piscine sono di area compresa #' tra 515.5 e 612.0 piedi^2, essendo rispettivamente il 1° e il 3° quantile #' e con maggior densita alineati alla retta qq della norma. #' Rimangono ancora degli outlier

readline("Press enter to continue: ")

#' Rimozione degli outlier attraverso intervallo di confidenza 95% per avvicinare alla norma mean_data = mean(imp_data) sd_data = sd(imp_data) confidence_level = 0.95 z_score = qnorm((1 + confidence_level) / 2)

lower_bound = mean_data - z_score * sd_data upper_bound = mean_data + z_score * sd_data

data_in_ci = imp_data[imp_data >= lower_bound & imp_data <= upper_bound] print(list(summary = summary(data_in_ci), Variance = var(data_in_ci), Standard_deviation = sd(data_in_ci))) boxplot(data_in_ci, horizontal = T, main = paste("Boxplot of PoolArea")) h = 0.9(min(sd(data_in_ci), IQR(data_in_ci))/1.34)NROW(data_in_ci)^(-1/5) br = (max(data_in_ci)-min(data_in_ci))/h print(paste("br_CI = ",br)) hist(data_in_ci, probability = T, col = 2, main = paste("Hist of PoolArea"), breaks = br) curve(dnorm(x, mean = mean(data_in_ci), sd = sd(data_in_ci)), add=T, col = 4) print(density(data_in_ci)) plot(density(data_in_ci, bw=h), main = "Density Distribution of PoolArea") qqnorm(data_in_ci, col = 3, main = paste("QQnorm of PoolArea")) qqline(data_in_ci, col = 5, pch = 10) #' Possiamo notare che la maggior parte delle piscine sono di area compresa #' tra 515.5 e 612.0 piedi^2, essendo rispettivamente il 1° e il 3° quantile #' e con maggior densita alineati alla retta qq della norma. } visualization_PoolArea(na.omit(House_prices[,"PoolArea"]), sum_info_h$PoolArea)

#' ----------------------------- #' 30: Visualizazione MiscVal #' -----------------------------

visualization_MiscVal <- function(data, sum_info_h){ par(mfrow = c(2,2)) print(sum_info_h) boxplot(data, horizontal = TRUE, main = "Boxplot of MiscVal") h = 0.9 * (min(sd(data), IQR(data)) / 1.34) * NROW(data)^(-1/5) br = (max(data) - min(data)) / h hist(data, probability = TRUE, col = 2, main = "Hist of MiscVal") curve(dnorm(x, mean = mean(data), sd = sd(data)), add = TRUE, col = 4) print(density(data)) plot(density(data), main = "Density Distribution of MiscVal") qqnorm(data, col = 3, main = "QQnorm of MiscVal") qqline(data, col = 5, pch = 10) #' Possiamo dedurre che la distribuzione è positivamente asimetrica. #' Si puo osservare che il costo delle funzionalità varie è prelevantemete #' 0 con 1453 occorrenze qundi non costruitto. #' Per fare un analisi suli costi delle funzionalità varie, serve ignorare quelli non

readline("Press enter to continue: ")

#' Rimozione dello 0 imp_data = data imp_data[imp_data == 0] = NA imp_data = na.omit(imp_data) print(list(summary = summary(imp_data), Variance = var(imp_data), Standard_deviation = sd(imp_data))) print(paste("len_data: ", length(data))) print(paste("len_data_w_0", length(imp_data))) print(paste("ratio_len_data_w_total", length(imp_data)/length(data)100)) boxplot(imp_data, horizontal = T, main = paste("Boxplot of MiscVal")) h = 0.9(min(sd(imp_data), IQR(imp_data))/1.34)*NROW(imp_data)^(-1/5) br = (max(imp_data)-min(imp_data))/h hist(imp_data, probability = F, col = 2, main = paste("Hist of MiscVal")) curve(dnorm(x, mean = mean(imp_data), sd = sd(imp_data)), add=T, col = 4) print(density(imp_data)) plot(density(imp_data), main = "Density Distribution of MiscVal") qqnorm(imp_data, col = 3, main = paste("QQnorm of MiscVal")) qqline(imp_data, col = 5, pch = 10) #' Possiamo notare che la maggior parte dei costi delle funzionalità varie #' sono di area compresa tra 515.5 e 612.0 piedi^2, #' essendo rispettivamente il 1° e il 3° quantile #' e con maggior densita alineati alla retta qq della norma. #' Dato la variabilita dei dati e la mancanza di coerenza, dovuta alla mancanza di dati #' (~=3.56%) dei dati totali (52 rispetto a 1460). Allora si potrebbe trascurare } visualization_MiscVal(na.omit(House_prices[,"MiscVal"]), sum_info_h$MiscVal)

#' ----------------------------- #' 31: Visualizazione MoSold #' -----------------------------

visualization_MoSold <- function(data, sum_info_h){ par(mfrow = c(1,1)) print("freq: ") cat("\n") print(table(data)) cat("\n") print("percintile: ") cat("\n") print(prop.table(table(data))*100) cat("\n") print(sum_info_h) barplot(table(data), col = 2, main = "Barplot of MoSold") #' Possiamo notare che i mesi con le moggior vendute sono fra il 5 e 8 mese con8 ~=55.69% #' Dal 1 si ha una crescita di vendute fino al 6 mese dove si il massimo di vendite con 253 #' vendite dopodicche la vendita decresce } visualization_MoSold(na.omit(House_prices[,"MoSold"]), sum_info_h$MoSold)

#' ----------------------------- #' 32: Visualizazione YrSold #' -----------------------------

visualization_YrSold <- function(data, sum_info_h){ par(mfrow = c(1,1)) print("freq: ") cat("\n") print(table(data)) cat("\n") print("percintile: ") cat("\n") print(prop.table(table(data))*100) barplot(table(data), col = 2, main = "Barplot of YrSold") #' Si puo notare che la vendita per anno dal 2006 a 2009 e per lo piu uniforme #' con l'anno di maggior vendita nel 2009 con 338 venidite #' Nel 2010 invece si ha il minimo di vendite con 175 vendite } visualization_YrSold(na.omit(House_prices[,"YrSold"]), sum_info_h$YrSold)

#' ----------------------------- #' 33: Visualizazione SalePrice #' -----------------------------

visualization_SalePrice <- function(data, sum_info_h){ par(mfrow = c(2,2)) print(sum_info_h) boxplot(data, horizontal = TRUE, main = "Boxplot of SalePrice") h = 0.9 * (min(sd(data), IQR(data)) / 1.34) * NROW(data)^(-1/5) br = (max(data) - min(data)) / h hist(data, probability = TRUE, col = 2, main = "Hist of SalePrice") curve(dnorm(x, mean = mean(data), sd = sd(data)), add = TRUE, col = 4) print(density(data)) plot(density(data), main = "Distribution of SalePrice with smoothing parameter bw") qqnorm(data, col = 3, main = "QQnorm of SalePrice") qqline(data, col = 5, pch = 10) #' Possiamo dedurre che la distribuzione è positivamente asimetrica #' e che la maggior parte del prezzo di vendita è compresa #' tra 129975 e 214000, essendo rispettivamente il 1° e il 3° quantile #' e con maggior densita alineati alla retta qq della norma. #' Rimangono ancora degli outlier

readline("Press enter to continue: ")

#' Rimozione degli outlier attraverso intervallo di confidenza 85% per avvicinare alla norma mean_data = mean(data) sd_data = sd(data) confidence_level = 0.85 z_score = qnorm((1 + confidence_level) / 2)

lower_bound = mean_data - z_score * sd_data upper_bound = mean_data + z_score * sd_data

data_in_ci = data[data >= lower_bound & data <= upper_bound] print(list(summary = summary(data_in_ci), Variance = var(data_in_ci), Standard_deviation = sd(data_in_ci))) boxplot(data_in_ci, horizontal = T, main = paste("Boxplot of PoolArea")) h = 0.9(min(sd(data_in_ci), IQR(data_in_ci))/1.34)NROW(data_in_ci)^(-1/5) br = (max(data_in_ci)-min(data_in_ci))/h print(paste("br_CI = ",br)) hist(data_in_ci, probability = T, col = 2, main = paste("Hist of PoolArea"), breaks = br) curve(dnorm(x, mean = mean(data_in_ci), sd = sd(data_in_ci)), add=T, col = 4) print(density(data_in_ci)) plot(density(data_in_ci, bw=h), main = "Density Distribution of PoolArea") qqnorm(data_in_ci, col = 3, main = paste("QQnorm of PoolArea")) qqline(data_in_ci, col = 5, pch = 10) #' Possiamo notare che la maggior parte del prezzo di vendita sono compresa #' tra 128988 e 196500, essendo rispettivamente il 1° e il 3° quantile #' e con maggior densita alineati alla retta qq della norma. } visualization_SalePrice(na.omit(House_prices[,"SalePrice"]), sum_info_h$SalePrice)
\end{lstlisting}
\end{center}
\end{document}