\documentclass{article}
\usepackage{amsmath, amssymb}

\title{Summary of Mathematical Methods for Artificial Intelligence}
\author{Stefano Scrobogna}
\date{June 2024}

\begin{document}
\maketitle

\section{Complex Numbers}
Complex numbers arise from the need to solve polynomial equations like \(z^2 = -1\). These equations do not have solutions in the real numbers, so complex numbers are introduced as an extension.

\subsection{Definition}
A complex number \(z\) is defined as:
\[
z = x + iy, \quad x, y \in \mathbb{R}
\]
where \(i\) is the imaginary unit satisfying \(i^2 = -1\). The set of complex numbers is denoted by \(\mathbb{C}\).

\subsection{Properties}
For a complex number \(z = x + iy\):
\begin{itemize}
    \item The real part of \(z\) is \(x\) and is denoted as \(\Re(z)\).
    \item The imaginary part of \(z\) is \(y\) and is denoted as \(\Im(z)\).
    \item The complex conjugate of \(z\) is \(\overline{z} = x - iy\).
    \item The modulus of \(z\) is \(|z| = \sqrt{x^2 + y^2}\).
\end{itemize}

\subsection{Trigonometric Form}
A complex number can be represented in trigonometric form:
\[
z = \rho (\cos \theta + i \sin \theta)
\]
where \(\rho = |z|\) is the modulus and \(\theta\) is the argument of \(z\), \(\theta = \text{atan2}(y, x)\).

\subsection{Exponential Form}
Using Euler's formula, a complex number can be expressed as:
\[
z = \rho e^{i\theta}
\]
where \(e^{i\theta} = \cos \theta + i \sin \theta\).

\section{Fourier Series}
Fourier series allow periodic functions to be expressed as a sum of sines and cosines.

\subsection{Definition}
A function \(f(x)\) defined on \([-L, L]\) can be expanded in a Fourier series:
\[
f(x) = \sum_{n=-\infty}^{\infty} c_n e^{i n \pi x / L}
\]
where \(c_n\) are the Fourier coefficients given by:
\[
c_n = \frac{1}{2L} \int_{-L}^{L} f(x) e^{-i n \pi x / L} \, dx
\]

\subsection{Convergence}
The pointwise convergence of the Fourier series is governed by the Dirichlet conditions, and the series converges to \(f(x)\) at points where \(f\) is continuous.

\section{Fourier Transform}
The Fourier transform generalizes the Fourier series to non-periodic functions.

\subsection{Definition}
The Fourier transform of a function \(f(x)\) is given by:
\[
\mathcal{F}\{f(x)\} = F(k) = \int_{-\infty}^{\infty} f(x) e^{-2 \pi i k x} \, dx
\]

\subsection{Inverse Fourier Transform}
The inverse Fourier transform is given by:
\[
f(x) = \mathcal{F}^{-1}\{F(k)\} = \int_{-\infty}^{\infty} F(k) e^{2 \pi i k x} \, dk
\]

\section{Lebesgue Integration}
Lebesgue integration is a fundamental concept in functional analysis and provides a more general framework than Riemann integration.

\subsection{Lebesgue Spaces}
Lebesgue spaces \(L^p\) are spaces of functions for which the \(p\)-th power of the absolute value is integrable:
\[
L^p(\mathbb{R}) = \left\{ f : \mathbb{R} \to \mathbb{R} \mid \int_{\mathbb{R}} |f(x)|^p \, dx < \infty \right\}
\]

\end{document}
