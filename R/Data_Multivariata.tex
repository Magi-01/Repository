\documentclass{article}
\usepackage{amsmath}
\usepackage{graphicx}
\usepackage{hyperref}

\title{Analisi dei Dati Multivariata}
\author{Nicola Torelli}
\date{2024}

\begin{document}

\maketitle

\section{Introduzione}
L'analisi multivariata riguarda l'analisi di un insieme di variabili \( x_1, x_2, \ldots, x_p \) (con \( p \geq 3 \)) misurate su \( n \) unità. Tale analisi permette di cogliere relazioni complesse presenti nei dati, che potrebbero non emergere dall'analisi di coppie di variabili.

Le variabili possono essere quantitative o categoriali. Nel caso di dati misti, è importante distinguere le variabili categoriali, definendole come fattori in R.

\section{Strumenti di Analisi}
\subsection{Regressione Multipla}
L'estensione dell'analisi di regressione semplice a quella multipla coinvolge una variabile risposta quantitativa e più variabili esplicative. Questa tecnica, fondamentale nell'analisi statistica, è ripresa in altri corsi, come il machine learning e i modelli statistici.

\subsection{Cluster Analysis}
L'analisi di raggruppamento o cluster analysis cerca pattern nei dati multivariati, identificando unità simili. Questo approccio, tipico dell'apprendimento non supervisionato, non prevede una variabile risposta.

\section{Analisi di Variabili Categoriali}
\subsection{Associazione Marginale e Condizionale}
L'associazione marginale tra due variabili categoriali può differire dall'associazione condizionale, come illustrato dal paradosso di Simpson. Un esempio sono i dati delle ammissioni ai dipartimenti dell'università di Berkeley nel 1973.

\section{Analisi di Variabili Quantitative}
\subsection{Matrice di Varianza-Covarianza e di Correlazione}
Per più variabili quantitative, si calcolano le covarianze o i coefficienti di correlazione lineare, organizzati in matrici simmetriche. Esempi includono i dati \textit{iris} e \textit{Cars93}, visualizzati tramite la funzione \texttt{ggcorrplot}.

\subsection{Scatterplot Matrix}
La funzione \texttt{pairs()} rappresenta tutti gli scatterplot delle coppie di variabili. Per dati misti, la funzione \texttt{ggpairs} del pacchetto GGally di ggplot adotta la rappresentazione grafica più appropriata.

\section{Regressione Lineare Multipla}
Nei modelli di regressione lineare multipla, una variabile risposta quantitativa \( Y \) è modellata come combinazione lineare di più variabili esplicative \( X_1, X_2, \ldots, X_p \). La funzione di regressione ha la forma:
\[ M(Y | x_1, x_2, \ldots, x_p) = \beta_0 + \beta_1 x_1 + \beta_2 x_2 + \cdots + \beta_p x_p \]

\end{document}
