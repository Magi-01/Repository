\documentclass[a4paper,12pt]{report}

% Packages
\usepackage{amsmath, amssymb, amsthm}
\usepackage{graphicx}
\usepackage{algorithm}
\usepackage{algorithmic}
\usepackage{hyperref}
\usepackage{geometry}
\geometry{margin=1in}

\begin{document}

% -------------------
% Title Page
% -------------------
\title{Hybrid Constraint Satisfaction and Markov Decision Process Framework for Delivery Scheduling under Uncertainty}
\author{Your Name}
\date{\today}
\maketitle

% -------------------
% Preface & Acknowledgement
% -------------------
\chapter*{Preface and Acknowledgements}
This thesis is submitted in partial fulfillment of the requirements for the degree of \dots \\
I would like to thank \dots

% -------------------
% Abstract
% -------------------
\chapter*{Abstract}
This thesis investigates the optimization of delivery scheduling in uncertain environments 
by integrating Constraint Satisfaction Problems (CSPs) and Markov Decision Processes (MDPs). 
The hybrid approach aims to minimize lateness, reduce resource usage, and maintain feasibility 
with respect to hard operational constraints. CSPs are used to enforce strict scheduling 
constraints such as capacity limits and delivery time windows, while MDPs capture stochastic 
elements such as uncertain travel times and dynamic order arrivals. The proposed hybrid 
framework applies exact Value Iteration on small-scale toy scenarios and employs 
Monte Carlo rollout methods for large-scale delivery simulations. Experimental evaluation 
compares baseline CSP-only methods with the hybrid approach, demonstrating improved robustness 
and adaptability under uncertainty.

% -------------------
% Table of Contents
% -------------------
\tableofcontents

% -------------------
% Introduction
% -------------------
\chapter{Introduction}
Efficient delivery scheduling is a central challenge in logistics and warehouse operations. 
Traditional optimization approaches often assume deterministic travel times and static order sets, 
yet real-world systems are inherently uncertain. Factors such as traffic conditions, 
vehicle breakdowns, and last-minute orders make robust scheduling difficult. 

This thesis addresses the problem by combining two complementary approaches:
\begin{itemize}
    \item Constraint Satisfaction Problems (CSPs) to enforce hard scheduling constraints.
    \item Markov Decision Processes (MDPs) to model stochastic decision-making under uncertainty.
\end{itemize}

The key research question is: \emph{How can a hybrid CSP--MDP framework improve 
delivery scheduling performance under uncertain conditions?}

% -------------------
% Literature Review
% -------------------
\chapter{Literature Review}
Relevant work spans three domains: 
\begin{enumerate}
    \item CSP-based scheduling and routing methods (e.g., vehicle routing with time windows).
    \item Markov Decision Process methods for uncertain planning.
    \item Hybrid AI methods that combine deterministic optimization with probabilistic models.
\end{enumerate}

This thesis builds on these ideas by applying them specifically to delivery systems.

% -------------------
% Methodology
% -------------------
\chapter{Methodology}

\section{Constraint Satisfaction Model}
We define binary assignment variables:
\[
x_{v,i} =
\begin{cases}
1 & \text{if vehicle $v$ serves order $i$}, \\
0 & \text{otherwise.}
\end{cases}
\]

Constraints:
\begin{align}
\sum_{v} x_{v,i} &= 1 \quad \forall i \in \mathcal{C} \quad \text{(each order assigned)} \\
\sum_{i \in \sigma_v} q_i &\le Q_v \quad \forall v \quad \text{(capacity)} \\
a_i \le t_i &\le b_i - s_i \quad \forall i \quad \text{(time windows)}
\end{align}

Objective function:
\[
\min \; \sum_{i} w^{(L)} \max(0, t_i + s_i - d_i) 
+ w^{(D)} \sum_{v} \text{dist}(\sigma_v) 
+ w^{(R)} \cdot \#\text{vehicles}
\]

\section{Markov Decision Process Model}
An MDP is defined by the tuple 
\((S, A, P, R, \gamma)\):
\begin{itemize}
    \item States \(s \in S\): vehicle positions, pending orders, current time.
    \item Actions \(a \in A(s)\): continue, reroute, reassign, wait, drop order.
    \item Transitions: 
    \[
    P(s'|s,a) = \Pr(\text{next state } s' \mid s, a)
    \]
    \item Reward function:
    \[
    R(s,a) = - \left( \alpha \cdot \text{lateness} 
    + \beta \cdot \text{travel cost}
    + \gamma \cdot \mathbf{1}_{\text{dropped}} \right)
    \]
\end{itemize}


\section{Hybridization Approaches}
Three approaches are considered:
\begin{enumerate}
    \item \textbf{Sequential Hybrid}: CSP generates a baseline plan, MDP handles local disruptions.
    \item \textbf{Coupled Iterative Hybrid}: MDP simulates futures, CSP reoptimizes with sampled scenarios.
    \item \textbf{Constraint-Guided Policy}: MDP proposes actions, CSP filters infeasible ones.
\end{enumerate}

\section{Value Iteration for Small Instances}
The Bellman optimality equation:
\[
V^*(s) = \max_{a \in A(s)} \Big[ R(s,a) + \gamma \sum_{s'} P(s'|s,a) V^*(s') \Big]
\]

\begin{algorithm}[h!]
\caption{Value Iteration}
\begin{algorithmic}
\STATE Initialize $V(s)=0$ for all states $s$
\REPEAT
    \STATE $\Delta \gets 0$
    \FOR{each state $s$}
        \STATE $v \gets V(s)$
        \STATE $V(s) \gets \max_{a \in A(s)} 
        \Big( R(s,a) + \gamma \sum_{s'} P(s'|s,a) V(s') \Big)$
        \STATE $\Delta \gets \max(\Delta, |v - V(s)|)$
    \ENDFOR
\UNTIL{$\Delta < \theta$}
\STATE Derive policy: $\pi(s) = \arg\max_{a \in A(s)} [ \cdots ]$
\end{algorithmic}
\end{algorithm}

\section{Monte Carlo Rollout for Large Instances}
For large-scale delivery, exact value iteration is infeasible.
Monte Carlo rollout approximates the Bellman update by sampling:

\[
Q(s,a) \approx \frac{1}{N} \sum_{i=1}^N 
\left( R(s,a) + \gamma V(s'_i) \right)
\]

where $s'_i$ are sampled next states under $(s,a)$.

---

% -------------------
% Results
% -------------------
\chapter{Results}
\begin{itemize}
    \item \textbf{Toy Case Study}: Exact Value Iteration for 1 vehicle, 3 deliveries.
    \item \textbf{Large Simulation}: Monte Carlo rollout with 20--50 deliveries.
    \item Compare baseline CSP-only vs hybrid CSP+MDP.
\end{itemize}

% -------------------
% Discussion
% -------------------
\chapter{Discussion}
The sequential hybrid was computationally efficient but limited in adaptability.
Monte Carlo rollout enabled scalable decision-making with reduced lateness, 
though at the cost of increased runtime.

% -------------------
% Conclusion
% -------------------
\chapter{Conclusion}
This thesis demonstrated a hybrid CSP--MDP approach to delivery scheduling under uncertainty. 
Exact value iteration provided a benchmark on small cases, while Monte Carlo rollout enabled 
scalable solutions. Future work includes integrating reinforcement learning and more 
sophisticated stochastic models.

% -------------------
% References
% -------------------
\chapter*{References}
\bibliographystyle{plain}
\bibliography{references}

\end{document}